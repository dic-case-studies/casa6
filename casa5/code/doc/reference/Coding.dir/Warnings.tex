\chapter{Programming Tips\label{Coding.Tips}}
\section{Identifiers}
Avoid identifiers which begin with one or two underscores, "\_" or
"\_\_".

\section{Classes}
A virtual destructor will be defined for a base class when {\em ALL} of the
\index{Virtual Destructor}
following hold:
\footnote{Plum, Saks; C++ Programming Guidelines; Plum Hall, 1991. (p. 208)}
\begin{itemize}
\item
The destructor in a derived class is different from the base class
destructor
\item
Derived class objects may be deleted via a pointer or reference to
base class objects.
\end{itemize}
However, the designer may have reasons for making the destructor
virtual when these do not hold.


Use copy constructors sparingly to avoid unecessary overhead.

Watch out for calls by reference and calls by value.

\section{Type Conversions}
\index{type conversions}
Explicit type conversions ("{\em cast}"s) are to be avoided except in the
\index{cast}
following cases:
\begin{itemize}
\item
Conversion of a pointer to a base class to a pointer of a derived
class within a type-safe container class
\item
Conversion of an external data representation (bit-stream, packed
message, etc.) into an object.  (Where possible use, RTTI and/or
previously defined I/O methods).
\end{itemize}
"{\em cast}"s for reasons other than those mentioned should be clearly
commented.

Avoid dependence on {\it implicit type conversions} -- those that are
performed on function arguments by the compiler.


\section{Portability}
\subsection{Data Representation/Byte Ordering}
Character constants should only contain one character because of
differences in portability; a string literal may be used instead.

Avoid hard-coding offsets into structures.  The alignment
of structure members is implementation dependent.  The pre-defined
"{\tt offsetof}" macro should be used to determine structure member
offset.
\ref{Warnings:Example Warnings.1}


\subsection{Execution Order}
Avoid dependence on initialization order of global static objects.

Avoid dependence on order of operand evaluation.

Avoid Dependence on initialization order in constructors.
\ref{Warnings:Example Warnings.2}


\section{Examples -- Warnings}

\index{warnings example 1}
\begin{verbatim}

#include <stddef.h>

   ....

   struct MyStruct {
      int x;
      int y;
   };


   ...

   int theYOffset = offsetof(MyStruct, y);

   ...

\end{verbatim}

\index{warnings example 2}
\begin{verbatim}

class A {
public:
   A() : y(9), x(7+y) {};   // NO !!! y may not be initialized before x
private:
   int x;
   int y;
}; // A

\end{verbatim}

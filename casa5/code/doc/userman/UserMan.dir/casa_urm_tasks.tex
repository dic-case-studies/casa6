%%%%%%%%%%%%%%%%%%%%%%%%%%%%%%%%%%%%%%%%%%%%%%%%%%%%%%%%%%%%%%%%%
%%%%%%%%%%%%%%%%%%%%%%%%%%%%%%%%%%%%%%%%%%%%%%%%%%%%%%%%%%%%%%%%%
%%%%%%%%%%%%%%%%%%%%%%%%%%%%%%%%%%%%%%%%%%%%%%%%%%%%%%%%%%%%%%%%%

% STM 2007-04-13  split from previous version
% STM 2007-04-19  bring over from cookbook appendix

\chapter{CASA Tasks}
\label{chapter:tasks}

%%%%%%%%%%%%%%%%%%%%%%%%%%%%%%%%%%%%%%%%%%%%%%%%%%%%%%%%%%%%%%%%%
%%%%%%%%%%%%%%%%%%%%%%%%%%%%%%%%%%%%%%%%%%%%%%%%%%%%%%%%%%%%%%%%%
\section{Using CASA Tasks}
\label{section:tasks.using}

There are currently a {\bf very} limited selection of tasks within
CASA which provide a streamlined interface to a subset of the CASA
applications; this subset should cover the basic reduction sequence
for continuum or spectral line analysis.

Please see the CASA Analysis Cookbook for the Toolkit for more
extensive information on the full suite of functionality.

While running CASA, you will have access to and be interacting with
tasks, either indirectly by providing parameters to a task, or
directly by running a task.  Each task has a well defined purpose, and
a number of associated parameters, the values of which are to be
supplied by the user.  Technically speaking, tasks are built on top of
tools - when you are running a task, you are running tools in the
toolkit, though this should be transparent.

To see what tasks are available in CASA, type {\tt tasklist}, e.g. 
\small
\begin{verbatim}
  CASA <1>: tasklist
  ---------> tasklist()
  Available tasks: 

  Import/Export   Information     Editing         Display/Plot
  -------------   -----------     -------         ------------
  importvla       listhistory     flagautocorr    plotants
  importasdm      listobs         flagger         plotcal 
  importfits                      flagxy          plotxy
  importuvfits                                    viewer
  exportfits
  exportuvfits

  Calibration     Imaging         Modelling       Utility
  -----------     -------         ---------       -------
  accum            clean           contsub         help task
  bandpass (B)    feather         setjy           help par.parameter
  blcal (M,MF)    ft              uvmodelfit      taskhelp
  correct         invert                          tasklist
  gaincal (G)     makemask                        browsetable
  fluxscale       mosaic                          split
  fringecal (K)                                   restore
  clearcal                                                
  pointcal 
  smooth

  Image Analysis  Simulation
  --------------  ----------
  imcontsub       almasim
  regrid
\end{verbatim}
\normalsize

Typing {\tt taskhelp} provides a one line description of all available 
tasks, e.g.
\small
\begin{verbatim}
  CASA <2>: taskhelp
  ---------> taskhelp()
  Available tasks: 

  accum        : Accumulate calibration solutions into a cumulative table
  bandpass     : Calculate a bandpass calibration solution
  blcal        : Calculate a baseline-based calibration solution
  browsetable  : Browse a visibility data set or calibration table
  clean        : Calculate a deconvolved image with selected clean algorithm
  clearcal     : Re-initialize visibility data set calibration data
  contsub      : Continuum fitting and subtraction in the uv plane
  correct      : Apply calculated calibration solutions
  exportuvfits : Export MS to UVFITS file
  feather      : Feather together an interferometer & a single dish image in the Fourier plane
  flagautocorr : Flag autocorrelations (typically in a filled VLA data set)
  fluxscale    : Bootstrap the flux density scale from standard calibraters
  fringecal    : Calculate a baseline-based fringe-fitting solution (phase, delay, delay-rate)
  ft           : Fourier transform the specified model (or component list)
  gaincal      : Calculate gain calibration solutions
  importvla    : Convert VLA archive file(s) to a CASA visibility data set (MS)
  importasdm   : Convert an ALMA Science Data Model directory to a CASA visibility data set (MS)
  importfits   : Convert a FITS image to a CASA image
  importuvfits : Convert a UVFITS file to a CASA visibility data set (MS)
  imhead       : Print image header to logger
  invert       : Calculate a dirty image and dirty beam
  listhistory  : List the processing history of a data set
  listobs      : List the observations in a data set
  flagxy       : Plot points for flagging selected X and Y axes
  makemask     : Calculate mask from image or visibility data set
  mosaic       : Calculate a multi-field deconvolved image with selected clean algorithm
  plotants     : Plot the antenna distribution in local reference frame
  plotcal      : Plot calibration solutions
  plotxy       : Plot points for selected X and Y axes
  pointcal     : Calculate pointing error calibration
  setjy        : Compute the model visibility for a specified source flux density
  split        : Create a new data set (MS) from a subset of an existing data set (MS)
  smooth       : Produce a smoothed calibration table
  uvmodelfit   : Fit a single component source model to the uv data
  viewer       : View an image or visibility data set
\end{verbatim}
\normalsize

Use the in-line help at the casapy prompt (e.g. {\tt help <taskname>}
to get the most up-to-date list of parameters.

Detailed information for each task are now given in the following
sections, alpabetized by task name.  

%%%%%%%%%%%%%%%%%%%%%%%%%%%%%%%%%%%%%%%%%%%%%%%%%%%%%%%%%%%%%%%%%
\section{{\tt accum}}
\label{section:tasks.accum}

\small
\begin{verbatim}
    Accumulate incremental calibration solutions into a cumulative calibration table:
    
    Keyword arguments:
    vis -- Name of input visibility file (MS)
            default: <unset>; example: vis='ngc5921.ms'
    tablein -- Input (cumulative) calibration table 
            default: '' (create); example: tablein='ngc5921.gcal'
    incrtable -- Input incremental calibration table name
            default: <unset>; example: incrtable='ngc5921_phase.gcal'
            <Note: caltable(out) = incrtable(in)*tablein(in)>
    caltable -- Output calibration table (cumulative)
            default: <unset>; example: caltable='ngc5921.Cgcal'
    field -- List of field names to update in input cumulative table 
            default: -1 (all); example: field='0957+561'
            <Note: field specifies names in tablein to which the incremental solution
            should be applied.>
    calfield  -- List of field names in incremental table to use.
            default: -1 (all); example: calfield='0917+624'
            <Note: calname specifies the names to select from incrtable to use when applying
            to tablein.>
    interp -- Interpolation mode to use on incremental solutions
            default: 'linear'; example: interp='nearest'
            <Options: 'linear','nearest','aipslin'>
    accumtime -- Cumulative table timescale when creating from scratch
            default: -1.0; example: accumtime=5.0
    spwmap -- Spectral windows to apply
            default: [-1]; example: spwmap=[0,0,1,1]
\end{verbatim}
\normalsize

%%%%%%%%%%%%%%%%%%%%%%%%%%%%%%%%%%%%%%%%%%%%%%%%%%%%%%%%%%%%%%%%%
\subsection{Setting Parameters and Invoking Tasks}
\label{section:intro.tasks.setpar}

Tasks require input parameters (sometimes called keywords).  One can
either 1) assign parameters, one at a time, from the CASA prompt, and
then execute the task; 2) execute the task in one line, specifically
assigning each parameter within the line; or 3) using the position
within the task call to specify the value of the parameter.  For
example the following are equivalent:

\small
\begin{verbatim}
  # Specify parameter names for each keyword input: 
    plotxy(vis='ngc5921.ms',xaxis='channel',yaxis='amp',datacolumn='data')
  # when specifying the parameter name, order doesn't matter, e.g.:
    plotxy(xaxis='channel',vis='ngc5921.ms',datacolumn='data',yaxis='amp')
  # use parameter order for invoking tasks
    plotxy('ngc5921.ms','channel','amp','data')
\end{verbatim}
\normalsize

You can also set parameters by performing the assigment within the
CASA shell and then inspecting them using the {\tt inp} command. 

\small
\begin{verbatim}
  CASA <5>: inp listobs         # type this to see the inputs needed for the listobs task
  --------> inp(listobs)        # CASA tells you that it is interpreting your input into
                                #  the correct format with parentheses.  

  listobs -- List the observations in a data set: # This is what is returned: 
  vis         = "False"         # A list of input parameters and what they are set to
  verbose     =  False          # If you have not set a parameter, then the default is shown. 

  CASA <6>: vis='ngc5921.ms'    # Set the parameter vis to be the visibility dataset ngc5921.ms
  CASA <7>: inp listobs         # Now ask to see the input list for the listobs task
  --------> inp(listobs)

  listobs -- List the observations in a data set:
  vis         = "ngc5921.ms"    # Now the keyword vis is set to the MS file you selected.  
  verbose     =  False

  CASA <8>: listobs             # now run the task with these inputs to get a summary list
                                # of the observation 
\end{verbatim}
\normalsize

You can also do the following: 
\small
\begin{verbatim}
  taskname=listobs              # define the task name to be listobs
  inp                           # now get a list of the inputs in that task. 
\end{verbatim}
\normalsize


All task parameters have global scope within CASA, i.e., the
parameter values are common to all tasks and at the CASA command
line. This allows the convenience of not changing parameters that are
shared between tasks but does require care when chaining together
sequences of task invocations (to insure proper values are provided).

If you want to reset all input keywords for all tasks, type:

\small
\begin{verbatim}
  CASA <10>: restore
\end{verbatim}
\normalsize

If you want to reset the input keywords for a single task, say clean, type:

\small
\begin{verbatim}
  CASA <12>: default('clean')
\end{verbatim}
\normalsize

To inspect a single parameter value just type it at the command line:
\small
\begin{verbatim}
  CASA <16>: alg          # type 'alg' to see the what the algorithm keyword is set to
    Out[16]: 'clark'      # CASA tells you it is set to use the Clark algorithm
\end{verbatim}
\normalsize

Help on a specific parameter can be obtained by typing 
{\tt help par.<parameter\_name>}, e.g.,

\small
\begin{verbatim}
  CASA <17>: help par.alg   # Ask for help on parameter 'alg'
  ---------> help(par.alg)

  Help on function alg in module parameter_dictionary: # CASA returns this: 
  alg()
    Imaging algorithm to use. Options: 'clark', 'hogbom', 'csclean', 'multiscale',
    'mfclark', 'mfhogbom', 'mfmultiscale':
    Default: 'hogbom'
\end{verbatim}
\normalsize


%%%%%%%%%%%%%%%%%%%%%%%%%%%%%%%%%%%%%%%%%%%%%%%%%%%%%%%%%%%%%%%%%
\subsection{Saving and Recalling Task Parameters}
\label{section:intro.tasks.parms}

Parameters for a given task can be saved by using the {\tt saveinputs}
command and restored using the {\tt execfile '<filename>'} command.  For
example:

\small
\begin{verbatim}
  CASA <1>: vis='ngc5921.ms'          # set the parameter vis to ngc5921.ms

  CASA <2>: inp 'listobs'             # See the inputs to the listobs task
  --------> inp('listobs')

  listobs -- List the observations in a data set:
  vis         = "ngc5921.ms"          # yep, its set.  
  verbose     =  False

  CASA <3>: saveinputs 'listobs'      # save listobs inputs to a file on disk called 
  --------> saveinputs('listobs')     # 'listobs.saved'

  CASA <4>: !more 'listobs.saved'     # view the listobs.saved file on disk.  

  vis         = "ngc5921.ms"          # yep, its here and the inputs are all nicely saved.  
  verbose     =  False

  CASA <5>: vis='someotherfile.ms'    # Now, set the keyword vis to someotherfile.ms

  CASA <6>: inp 'listobs'             # See the inputs to the listobs task
  --------> inp('listobs')

  listobs -- List the observations in a data set:
  vis         = "someotherfile.ms"    # OK, vis is set to this new file name
  verbose     =  False

  CASA <7>: execfile 'listobs.saved'  # now execute the listobs.saved file to restore the 
  --------> execfile('listobs.saved') # inputs to the previous values. 

  CASA <8>: inp 'listobs'             # Look at the listobs inputs now
  --------> inp('listobs')

  listobs -- List the observations in a data set:
  vis         = "ngc5921.ms"          # Yep, they are back to the original values that you 
  verbose     =  False                # saved before.  

  CASA <9>: saveinputs 'listobs','ngc5921_listobs.par'  # Now save parameters to a file with 
  --------> saveinputs('listobs','ngc5921_listobs.par') # a name that you choose

  CASA <10>: !more ngc5921_listobs.par  # Look at that file on disk
  vis         = "ngc5921.ms"            # you can run this anytime with the command 
  verbose     =  False                  # execfile ngc5921_listobs.par
\end{verbatim}
\normalsize


Your command line history is automatically maintained and stored in
the local directory as ipython.log; this file can be edited and
re-executed as appropriate using the {\tt execfile '<filename>'} feature.
In addition, logging of output from commands is sent to the file
casapy.log, also in your local directory; this is brought up
automatically.

The logger has a range of features including the ability to filter
messages, sort by Priority, Time, etc, and the ability to insert
additional comments. The CASA logger is shown in Figures 1.1 - 1.4.  

\begin{figure}[ht]
\gname{casalogger1}{6}
\caption{\label{fig:logger} CASA Logger GUI window}
\hrulefill
\end{figure}

\begin{figure}[h]
\gname{casalogger_select}{6}
\caption{\label{fig:logger_search} CASA Logger - Search example:
Specify a string in the entry box to
have all instances of the found string highlighted.}
\hrulefill
\end{figure}

\begin{figure}[h]
\gname{casalogger_filter}{6}
\caption{\label{fig:logger_filter} CASA Logger - Filter facility: The
log output can be sorted by Priority, Time, Origin. One can also
filter for a string found in the Message.}  
\hrulefill
\end{figure}

\begin{figure}[h]
\gname{casalogger_insert}{6}
\caption{\label{fig:logger_insert} CASA Logger - Insert facility: The
log output can be augmented by adding notes or comments during the
reduction. The file should then be saved to disk to retain these
changes.}
\hrulefill
\end{figure}


%%%%%%%%%%%%%%%%%%%%%%%%%%%%%%%%%%%%%%%%%%%%%%%%%%%%%%%%%%%%%%%%%
\section{{\tt bandpass}}
\label{section:tasks.bandpass}

\small
\begin{verbatim}
    Calculate a bandpass calibration solution:
    
    Keyword arguments:
    vis -- Name of input visibility file (MS)
            default: <unset>; example: vis='ngc5921.ms'
    caltable -- Name of output Gain calibration table
            default: <unset>; example: caltable='ngc5921.gcal'
    mode -- Type of data selection
            default: 'none' (all data); example: mode='channel'
            <Options: 'channel', 'velocity', 'none'>
    nchan -- Number of channels to select (when mode='channel')
            default: 0; example: nchan=45 (note: nchan=0 selects all)
    start -- Start channel, 0-relative;active when mode='channel','velocity'
            default=0; example: start=5,start='20km/s'
    step -- Increment between channels;active when mode='channel','velocity'
            default=1; example: step=1, step='1km/s'
    msselect -- Optional subset of the data to select (field,spw,time,etc)
            default:'';example:msselect='FIELD_ID==0', 
            msselect='FIELD_ID IN [0,1]', msselect='FIELD_ID <= 1'
            <Options: see: http://aips2.nrao.edu/docs/notes/199/199.html>
    --- Solve properties
    solint --  Solution interval (sec)
            default: 86400 (=long time); example: solint=60.
    refant -- Reference antenna *name*
            default: -1 (none); example: refant='13'
    bandtype -- Type of bandpass solution (B or BPOLY)
            default: 'B'; example: bandtype='BPOLY'
    append -- Append solutions to the (existing) table
            default: False; example: append=True
    degamp -- Polynomial degree for amplitude solution
            default: 3; example: degamp=2
    degphase -- Polynomial degree for phase solution
            default: 3; example: degphase=2
    visnorm -- Normalize data prior to solution
            default: False; example: visnorm=True
    bpnorm -- Normalize result?
            default: True; example: bpnorm=False
    maskcenter -- Number of channels to avoid in center of each band
            default: 1; example: maskcenter=5
    maskedge -- Fraction of channels to avoid at each band edge (in %)
            default: 5; example: maskedge=3
    --- Other calibration to pre-apply (weights are calibrated by default)
    gaincurve -- Apply VLA antenna gain curve correction
            default: True; example: gaincurve=False
            <Options: True, False>
    opacity -- Apply opacity correction
            default: True; example: opacity=False
            <Options: True, False>
    tau -- Opacity value to apply (if opacity=True)
            default:0.0001; example: tau=0.0005
    gaintable -- Gain calibration solutions
            default: ''; example: gaintable='ngc5921.gcal'
    gainselect -- Select subset of calibration solutions from gaintable
            default:''; example: gainselect='FIELD_ID==1'
    bptable -- Bandpass calibration solutions
            default: ''; example: bptable='ngc5921.bcal'
    pointtable -- Pointing calibration solutions
            defalut: ''; example: pointtable='ngc5921.ptcal'
\end{verbatim}
\normalsize


%%%%%%%%%%%%%%%%%%%%%%%%%%%%%%%%%%%%%%%%%%%%%%%%%%%%%%%%%%%%%%%%%
\section{{\tt blcal}}
\label{section:tasks.blcal}

\small
\begin{verbatim}
    Calculate a baseline-based calibration solution (gain or bandpass):
    
    Keyword arguments:
    vis -- Name of input visibility file (MS)
            default: <unset>; example: vis='ngc5921.ms'
    caltable -- Name of output Gain calibration table
            default: <unset>; example: caltable='ngc5921.gcal'
    mode -- Type of data selection
            default: 'none' (all data); example: mode='channel'
            <Options: 'channel', 'velocity', 'none'>
    nchan -- Number of channels to select (when mode='channel')
            default: 1; example: nchan=45 (note: nchan=0 selects all)
    start -- Start channel, 0-relative;active when mode='channel','velocity'
            default=0; example: start=5,start='20km/s'
    step -- Increment between channels;active when mode='channel','velocity'
            default=1; example: step=1, step='1km/s'
    msselect -- Optional subset of the data to select (field,spw,time,etc)
            default:'';example:msselect='FIELD_ID==0', 
            msselect='FIELD_ID IN [0,1]', msselect='FIELD_ID <= 1'
            <Options: see: http://aips2.nrao.edu/docs/notes/199/199.html>
    gaincurve -- Apply VLA antenna gain curve correction
            default: True; example: gaincurve=False
            <Options: True, False>
    opacity -- Apply opacity correction
            default: True; example: opacity=False
            <Options: True, False>
    tau -- Opacity value to apply (if opacity=True)
            default:0.0001; example: tau=0.0005
    gaintable -- Gain calibration solutions
            default: ''; example: gaintable='ngc5921.gcal'
    gainselect -- Select subset of calibration solutions from gaintable
            default:''; example: gainselect='FIELD_ID==1'
    solint --  Solution interval (sec)
            default: 86400 (=long time); example: solint=60.
    refant -- Reference antenna *name*
            default: -1 (any integer=none); example: refant='13'
    freqdep -- Solve for frequency dependent solutions
            default: False (gain; True=bandpass); example: freqdep=True
\end{verbatim}
\normalsize


%%%%%%%%%%%%%%%%%%%%%%%%%%%%%%%%%%%%%%%%%%%%%%%%%%%%%%%%%%%%%%%%%
\section{{\tt browsetable}}
\label{section:tasks.browsetable}

\small
\begin{verbatim}
    Browse a table (visibility data set or calibration table):
    
    Keyword arguments:
    vis -- Name of input visibility file (MS)
            default: <unset>; example: vis='ngc5921.ms'

\end{verbatim}
\normalsize


%%%%%%%%%%%%%%%%%%%%%%%%%%%%%%%%%%%%%%%%%%%%%%%%%%%%%%%%%%%%%%%%%
\section{{\tt clean}}
\label{section:tasks.clean}

\small
\begin{verbatim}    
    Calculate a deconvolved image with selected clean algorithm:
    Keyword arguments:
    vis -- Name of input visibility file (MS)
            default: <unset>; example: vis='ngc5921.ms'
    imagename -- Name of output images: restored=imagename.restored;
            residual=imagename.residual,model=imagename
            default: <unset>; example: imagename='ngc5921'
    mode -- Type of selection 
            default: <unset>; example: mode='channel'; 
            <Options: 'mfs', channel, velocity'>
    alg -- Algorithm to use
            default: 'hogbom'; example: alg='clark'; 
            <Options: 'clark','hogbom','csclean','multiscale'>
    niter -- Number iterations, set to zero for no CLEANing
            default: 500; example: niter=500
    gain -- Loop gain for CLEANing
            default: 0.1; example: gain=0.1
    threshold -- Flux level at which to stop CLEANing (units=mJy)
            default: 0.0; example: threshold=0.0
    mask -- Name(s) of mask image(s) used for CLEANing
            default: <unset>; example: mask='orion.mask'
    cleanbox -- List of [blc-x,blc-y,trc-x,trc-y] values
            default: []; example: cleanbox=[110,110,150,145]
            Note: This can also be a filename with clean values:
            fieldindex blc-x blc-y trc-x trc-y
    nchan -- Number of channels to select
            default: 1; example: nchan=45
    start -- Start channel, 0-relative
            default=0; example: start=5
    width -- Channel width (value > 1 indicates channel averaging)
            default=1; example: width=5
    step -- Step in channel number
            default=1; example: step=2      
    imsize -- Image size in spatial pixels (x,y)
            default = [256,256]; example: imsize=[512,512]
    cell -- Cell size in arcseconds (x,y)
            default=[1,1]; example: cell=[0.5,0.5]
    stokes -- Stokes parameters to image
            default='I'; example: stokes='IQUV'; 
            <Options: 'I','IV','IQU','IQUV'>
    fieldid -- Field index identifier
            default=0; example: fieldid=1
    field -- Field name(s); this will use a minimum match on the strings
            that you provide; to use multiple strings, enter a single string
            with spaces separating the field names
            default: ''(will use fieldid); example: field='133',field='133 N5921'
    spwid -- Spectral window index identifier
            default=-1 (all); example: spwid=1
    weighting -- Weighting to apply to visibilities
            default='natural'; example: weighting='uniform'; 
            <Options: 'natural','uniform','briggs','radial', 'superuniform'>
    rmode -- Robustness mode (for use with Brigg's weighting)
            default='none'; example='abs'; 
            <Options: 'norm','abs','none'>
    robust -- Brigg's robustness parameter 
            default=0.0; example: robust=0.5; 
            <Options: -2.0 to 2.0; -2 (uniform)/+2 (natural)>
    uvfilter -- Apply additional filtering/uv tapering of the visibilities.
            defalt=False; example: uvfilter=True
    uvfilterbmaj -- Major axis of filter (arcseconds)
            default=1.; example: uvfilterbmaj=12.
    uvfilterbmin -- Minor axis of filter (arcseconds)
            default=1.; example: uvfilterbmin=12.
    uvfilterbpa -- Position angle of filter (degrees)
            default=0.' example: uvfilterbpa=57.
\end{verbatim}
\normalsize


%%%%%%%%%%%%%%%%%%%%%%%%%%%%%%%%%%%%%%%%%%%%%%%%%%%%%%%%%%%%%%%%%
\section{{\tt clearcal}}
\label{section:tasks.clearcal}

\small
\begin{verbatim}
    Re-initializes the calibration for a visibility data set;
    MODEL_DATA is set to unity (in total intensity 
    and unpolarized) and CORRECTED_DATA is set to the original (observed) DATA:
    
    Keyword arguments:
    vis -- Name of input visibility file (MS)
            default: <unset>; example: vis='ngc5921.ms'

\end{verbatim}
\normalsize


%%%%%%%%%%%%%%%%%%%%%%%%%%%%%%%%%%%%%%%%%%%%%%%%%%%%%%%%%%%%%%%%%
\section{{\tt contsub}}
\label{section:tasks.contsub}

\small
\begin{verbatim}
    Continuum fitting and subtraction in the uv plane:
    
    Keyword arguments:
    vis -- Name of input visibility file (MS)
            default: <unset>; example: vis='ngc5921.ms'
    fieldid -- Field index identifier
            default=unset; example: fieldid=1
    field -- Field name(s); this will use a minimum match on the strings
            that you provide; to use multiple strings, enter a single string
            with spaces separating the field names
            default: ''(will use fieldid); example: field='133',field='133 N5921'
    spwid -- Spectral window index identifier
            default=0; example: spwid=1
    channels -- Range of channels to fit
            default:; example: channels=range(4,7)+range(50,60)
    solint -- Averaging time (seconds)
            default: 0.0 (scan-based); example: solint=10
    fitorder -- Polynomial order for the fit
            default: 0; example: fitorder=1
    fitmode -- Use of the continuum fit model
            default: 'subtract'; example: fitmode='replace'
            <Options: 
            'subtract'-store continuum model and subtract from data,
            'replace'-replace vis with continuum model,
            'model'-only store continuum model>
\end{verbatim}
\normalsize


%%%%%%%%%%%%%%%%%%%%%%%%%%%%%%%%%%%%%%%%%%%%%%%%%%%%%%%%%%%%%%%%%
\section{{\tt correct}}
\label{section:tasks.correct}

\small
\begin{verbatim}
    Apply calibration solution(s) to data:
    
    Keyword arguments:
    vis -- Name of input visibility file (MS)
            default: <unset>; example: vis='ngc5921.ms'
    msselect -- Optional subset of the data to select (field,spw,time,etc)
            default:'';example:msselect='FIELD_ID==0', 
            msselect='FIELD_ID IN [0,1]', msselect='FIELD_ID <= 1'
            <Options: see: http://aips2.nrao.edu/docs/notes/199/199.html>
    gaincurve -- Apply VLA antenna gain curve correction
            default: True; example: gaincurve=False
            <Options: True, False>
    opacity -- Apply opacity correction
            default: True; example: opacity=False
            <Options: True, False>
    tau -- Opacity value to apply (if opacity=True)
            default:0.0001; example: tau=0.0005
    gaintable -- Gain calibration solutions
            default: ''; example: gaintable='ngc5921.gcal'
    gainselect -- Select subset of calibration solutions from gaintable
            default:''; example: gainselect='FIELD_ID==1'
    bptable -- Bandpass calibration solutions
            default: ''; example: bptable='ngc5921.bcal'
    blbased -- Apply baseline-based solutions (from blcal)
            default: False; example: blbased=True
    pointtable -- Pointing calibration solutions
            default: ''; example: pointtable='ngc5921.ptcal'
    calwt -- Calibrate weights along with data
            default: False; example: calwt=True
\end{verbatim}
\normalsize


%%%%%%%%%%%%%%%%%%%%%%%%%%%%%%%%%%%%%%%%%%%%%%%%%%%%%%%%%%%%%%%%%
\section{{\tt exportuvfits}}
\label{section:tasks.exportuvfits}

\small
\begin{verbatim}
    Convert a CASA visibility data set (MS) to a UVFITS file:
    The FITS file is always written in floating point format and the
    data is always stored in the primary array of the FITS file.
    By default, a single-source UVFITS file is written, but if the
    MS contains more than one field or if you select the multi-source
    argument=True, a multi-source UVFITS file will be written (you
    must ensure that the data shape is fixed for all data due to the
    limitations of the UVFITS format).
    
    Keyword arguments:
    vis -- Name of input visibility file (MS)
            default: <unset>; example: vis='3C273XC1.ms'
    fitsfile -- Name of output UV FITS file
            default: <unset>; example='3C273XC1.fits'
    datacolumn -- Data column to write
            default: 'corrected'; example: datacolumn='data'
            <Options: 'corrected','model','data'>
    fieldid -- Field index identifier
            default=0; example: fieldid=1
    field -- Field name(s); this will use a minimum match on the strings
            that you provide; to use multiple strings, enter a single string
            with spaces separating the field names
            default: ''(will use fieldid); example: field='133',field='133 N5921'
    spwid -- Spectral window index identifier
            default=0; example: spwid=1
    nchan -- Number of channels to select
            default: 1; example: nchan=45
    start -- Start channel, 0-relative
            default=0; example: start=5
    width -- Channel width (value > 1 indicates channel averaging)
            default=1; example: width=5
    writesyscal -- Write GC and TY tables
            default=False; example: writesyscal=True
    multisource -- Write in multi-source format
            default=False; example: multisource=True
    combinespw -- Handle spectral window as IF
            default=False; example: combinespw=True
    writestation -- Write station name instead of antenna name
            default=False; example: writestation=True
\end{verbatim}
\normalsize


%%%%%%%%%%%%%%%%%%%%%%%%%%%%%%%%%%%%%%%%%%%%%%%%%%%%%%%%%%%%%%%%%
\section{{\tt feather}}
\label{section:tasks.feather}

\small
\begin{verbatim}
    Feather together an interferometer and a single dish image in the Fourier plane: 
    
    Keyword arguments:
    imagename -- Name of output feathered image
            default: False; example: imagename='orion_combined'
    highres -- Name of high resolution (interferometer) image
            default: False; example: imagename='orion_vla.im'
    lowres -- Name of low resolution (single dish) image
            default: False; example: imagename='orion_gbt.im'

\end{verbatim}
\normalsize


%%%%%%%%%%%%%%%%%%%%%%%%%%%%%%%%%%%%%%%%%%%%%%%%%%%%%%%%%%%%%%%%%
\section{{\tt flagautocorr}}
\label{section:tasks.flagautocorr}

\small
\begin{verbatim}
    Flag autocorrelations (typically in a filled VLA data set):
    
    Keyword arguments:
    vis -- Name of input visibility file (MS)
            default: <unset>; example: vis='ngc5921.ms'

\end{verbatim}
\normalsize


%%%%%%%%%%%%%%%%%%%%%%%%%%%%%%%%%%%%%%%%%%%%%%%%%%%%%%%%%%%%%%%%%
\section{{\tt flagdata}}
\label{section:tasks.flagdata}

\small
\begin{verbatim}
    Flag data based on selections:
    
    Keyword arguments:
    vis -- Name of input visibility file (MS)
            default: <unset>; example: vis='ngc5921.ms'
    antennaid -- Antenna index identifier
            default: [-1] (all); example: antennaid=3
    baseline -- Baseline (composed of a list: [ant1,ant2])
            default: [-1] (all); example: baseline=[2,3]
    chans -- Channel range to clip
            default: <unset>; example: chans=range(0,4)+range(55,62)
            Note: Remember in Python ranges go up to but not including
            the last value (so the first range covers channels 0,1,2,3).
    clipfunction -- Defines the function used in evaluating data for clipping
            default: 'ABS'; example: clipfunction='RE'
            <Options: 'ABS','ARG','RE','IM','NORM'>
    clipcorr -- Defines the correlation(s) to clip
            default: 'I'; example: clipcorr='RR'
            <Options: 'I','XX','YY','RR','LL'>
    clipminmax -- Sets the range of data values that will not be flagged
            default: <unset>; example: clipminmax=[0.0,1.5]
    fieldid -- Field index identifier
            default: -1 (all); example: fieldid=1
    field -- Field name(s); this will use a minimum match on the strings
            that you provide; to use multiple strings, enter a single string
            with spaces separating the field names
            default: ''(will use fieldid); example: field='133',field='133 N5921'
    spwid -- Spectral window index identifier
            default: -1 (all); example: spwid=0
    timerange -- Time range selection
            default: <unset>; example: timerange=['26-APR-2003/02:45:00.0','26-APR-2003/02:49:00.0']
    unflag -- Unflag the data (True/False)
            default: False; example: unflag=True
\end{verbatim}
\normalsize


%%%%%%%%%%%%%%%%%%%%%%%%%%%%%%%%%%%%%%%%%%%%%%%%%%%%%%%%%%%%%%%%%
\section{{\tt flagxy}}
\label{section:tasks.flagxy}

\small
\begin{verbatim}
    Plot points for flagging selected X and Y axes; if region is not
    specified, it enables interactive setting of  
    flag boxes using the mouse. Use the task flagdata to write these flags to disk.
    
    Keyword arguments:
    vis -- Name of input visibility file (MS)
            default: <unset>; example: vis='ngc5921.ms'
    xaxis -- Visibility file (MS) data to plot along the x-axis
            <Options: 'azimuth','baseline','channel','elevation',
            'hourangle','parallacticangle','time','u','uvdist','x'>
    yaxis -- Visibility data to plot along the y-axis
            <Options: 'amp','phase','v'>
    datacolumn -- Visibility file (MS) data column.
            default: 'corrected'; example: datacolumn='model'
            <Options: 'data' (raw),'corrected','model','weight'>
    antennaid -- Antenna index identifier
            default: -1 (all); example: antennaid=[13]
    spwid -- Spectral window index identifier
            default: -1 (all); example: spwid=0
    fieldid -- Field index identifier
            default: -1 (all); example: fieldid=0
    field -- Field name(s); this will use a minimum match on the strings
            that you provide; to use multiple strings, enter a single string
            with spaces separating the field names
            default: ''(will use fieldid); example: field='133',field='133 N5921'
    timerange
    correlations -- Correlations to plot
            default: '' (all); example: correlations='RR'
            <Options: '','RR','LL','RR LL','XX','YY','XX YY'>
    --- Spectral Information ---
    nchan -- Number of channels to select
            default: -1 (all); example: nchan=55
    start -- Start channel (0-relative)
            default=0; example: start=5
    width -- Channel width (value>0 indicates averaging)
            default=1; example: width=5
    --- Plot Options ---
    subplot -- Panel number on the display screen
            default: 111 (full screen display); example:
               subplot=221 (upper left panel of 4 on frame)
               subplot=223 (lower left panel of 4 on frame)
               subplot=212 (lower panel of 2 on frame)
            <Options: subplot=yxn; where y=number of rows,
            x=number of columns, n=panel number (numbered starting
            at 1 from top left)>
    overplot -- Overplot these values on current plot (if possible)
            default: False; example: overplot=True
    plotsymbol -- pylab plot symbol
            default: ','; example: plotsymbol='bo' (blue circles)
            <Options: '-' = solid line
                      'o' = filled circles
                      '.' = points
                      '-.ro' = combined dash dot line with red circles
    title -- Plot title (above plot)
            default: ''; example: title='This is my title'
    xlabels -- Label for x axis
            default: ''; example: xlabels='X Axis'
    ylabels -- Label for y axis
            default: ''; example: ylabels='Y Axis'
    iteration -- Plot each value of this data axis on a separate plot
            default: '' (no iteration); example: iteration='antenna'
            <Options: 'antenna', 'baseline','time'>
    fontsize -- Font size for labels
            default: 1; example: fontsize=2
    windowsize -- Window size
            default: 1.0; example: windowsize=0.5
    --- Flag Options ---
    region -- Flagging region specified as [xmin,xmax,ymin,ymax]
            default: [0.0] = mark region interactively
            example: region=[53,60,0.0,2.0]
    diskwrite -- Write flags to disk or just display them
            Note: This parameter is only active when region is set.
            default: False; example: diskwrite=True
\end{verbatim}
\normalsize


%%%%%%%%%%%%%%%%%%%%%%%%%%%%%%%%%%%%%%%%%%%%%%%%%%%%%%%%%%%%%%%%%
\section{{\tt fluxscale}}
\label{section:tasks.fluxscale}

\small
\begin{verbatim}
    Bootstrap the flux density scale from standard calibrators:
    
    Keyword arguments:
    vis -- Name of input visibility file (MS)
            default: <unset>; example: vis='ngc5921.ms'
    caltable -- Name of input calibration table
            default: <unset>; example: caltable='ngc5921.gcal'
    fluxtable -- Name of output, flux-scaled calibration table
            default: <unset>; example: fluxtable='ngc5921.gcal2'
    reference -- Reference field name (transfer flux scale from)
            default: <unset>; example: reference='1328+307'
    transfer -- Transfer field name(s)
            default: -1 (all); example: transfer=['1445+09900002']
\end{verbatim}
\normalsize


%%%%%%%%%%%%%%%%%%%%%%%%%%%%%%%%%%%%%%%%%%%%%%%%%%%%%%%%%%%%%%%%%
\section{{\tt fringecal}}
\label{section:tasks.fringecal}

\small
\begin{verbatim}
    Calculate a baseline-based fringe-fitting solution (phase, delay,
    delay-rate); Note: currently this is limited  
    to delays and delay-rates in the central ambiguity:
    
    Keyword arguments:
    vis -- Name of input visibility file (MS)
            default: <unset>; example: vis='ngc5921.ms'
    caltable -- Name of output Gain calibration table
            default: <unset>; example: caltable='ngc5921.gcal'
    mode -- Type of data selection
            default: 'none' (all data); example: mode='channel'
            <Options: 'channel', 'velocity', 'none'>
    nchan -- Number of channels to select (when mode='channel')
            default: 1; example: nchan=45 (note: nchan=0 selects all)
    start -- Start channel, 0-relative;active when mode='channel','velocity'
            default=0; example: start=5,start='20km/s'
    step -- Increment between channels;active when mode='channel','velocity'
            default=1; example: step=1, step='1km/s'
    msselect -- Optional subset of the data to select (field,spw,time,etc)
            default:'';example:msselect='FIELD_ID==0', 
            msselect='FIELD_ID IN [0,1]', msselect='FIELD_ID <= 1'
            <Options: see: http://aips2.nrao.edu/docs/notes/199/199.html>
    gaincurve -- Apply VLA antenna gain curve correction
            default: False; example: gaincurve=False
            <Options: True, False>
    opacity -- Apply opacity correction
            default: False; example: opacity=False
            <Options: True, False>
    tau -- Opacity value to apply (if opacity=True)
            default:0.0001; example: tau=0.0005
    solint --  Solution interval (sec)
            default: 0; example: solint=60.
    refant -- Reference antenna *name*
            default: -1 (any integer=none); example: refant='13'
\end{verbatim}
\normalsize


%%%%%%%%%%%%%%%%%%%%%%%%%%%%%%%%%%%%%%%%%%%%%%%%%%%%%%%%%%%%%%%%%
\section{{\tt ft}}
\label{section:tasks.ft}

\small
\begin{verbatim}
    Fourier transform the specified model (or component list) and insert
    into the MODEL_DATA column of the specified visibility data set:  
    
    Keyword arguments:
    vis -- Name of input visibility file (MS)
            default: <unset>; example: vis='ngc5921.ms'
    model -- Name of input model image
            default: False; example: model='3C373XC1.model'
    complist -- Name of component list
            default: ''; example: complist='test.cl'
    incremental -- Add to the existing MODEL_DATA column?
            default: False; example: incremental=True
\end{verbatim}
\normalsize


%%%%%%%%%%%%%%%%%%%%%%%%%%%%%%%%%%%%%%%%%%%%%%%%%%%%%%%%%%%%%%%%%
\section{{\tt gaincal}}
\label{section:tasks.gaincal}

\small
\begin{verbatim}
    Solve for gain calibration:
    
    Keyword arguments:
    vis -- Name of input visibility file (MS)
            default: <unset>; example: vis='ngc5921.ms'
    caltable -- Name of output Gain calibration table
            default: <unset>; example: caltable='ngc5921.gcal'
    mode -- Type of data selection
            default: 'none' (all data); example: mode='channel'
            <Options: 'channel', 'velocity', 'none'
    nchan -- Number of channels to select (when mode='channel')
            default: 1; example: nchan=45 (note: nchan=0 selects all)
    start -- Start channel, 0-relative;active when mode='channel','velocity'
            default=0; example: start=5,start='20km/s'
    step -- Increment between channels;active when mode='channel','velocity'
            default=1; example: step=1, step='1km/s'
    msselect -- Optional subset of the data to select (field,spw,time,etc)
            default:'';example:msselect='FIELD_ID==0', 
            msselect='FIELD_ID IN [0,1]', msselect='FIELD_ID <= 1'
            <Options: see: http://aips2.nrao.edu/docs/notes/199/199.html>
    uvrange -- Optional subset of the uv range (kilo lambda)
            default=[0] (all); example: uvrange=[0,15]
    --- Solve properties
    solint --  Solution interval (sec)
            default: -1 (scan based); example: solint=60.
    refant -- Reference antenna *name*
            default: -1 (any integer=none); example: refant='13'
    gaintype -- Type of gain solution (G, T, or GSPLINE)
            default: 'G'; example: gaintype='GSPLINE'
    calmode -- Type of solution
            default: 'ap' (amp and phase); example: calmode='p'
            <Options: 'p','a','ap'>
    append -- Append solutions to the (existing) table
            default: False; example: append=True
    splinetime -- Spline timescale (sec); used for gaintype='GSPLINE'
            default: 10800; example: splinetime=5000
    npointaver -- Tune phase-unwrapping algorithm for gaintype='GSPLINE'
            default: 10; example: npointaver=5
    phasewrap -- Wrap the phase when differences greater than this are seen (degrees)
            Used for gaintype='GSPLINE'
            default: 250; example: phasewrap=200
    --- Other calibration to pre-apply (weights are calibrated by default)
    gaincurve -- Apply VLA antenna gain curve correction
            default: True; example: gaincurve=False
            <Options: True, False>
    opacity -- Apply opacity correction
            default: True; example: opacity=False
            <Options: True, False>
    tau -- Opacity value to apply (if opacity=True)
            default:0.0001; example: tau=0.0005
    gaintable -- Gain calibration solutions
            default: ''; example: gaintable='ngc5921_phase.gcal'
    bptable -- Bandpass calibration solutions
            default: ''; example: bptable='ngc5921.bcal'
    pointttable -- Pointing calibration solutions
            default: ''; example: pointtable='ngc5921.ptcal'
\end{verbatim}
\normalsize


%%%%%%%%%%%%%%%%%%%%%%%%%%%%%%%%%%%%%%%%%%%%%%%%%%%%%%%%%%%%%%%%%
\section{{\tt importarchive}}
\label{section:tasks.importarchive}

\small
\begin{verbatim}
    Convert VLA archive file(s) to a CASA visibility data set (MS):
    
    Keyword arguments:
    archivefiles -- Name of input VLA archive file(s)
            default: <unset>
            example: archivefiles=['AP314_A950519.xp1','AP314_A950519.xp2']
    vis -- Name of output visibility file (MS)
            default: <unset>; example: vis='NGC7538.ms'
    bandname -- VLA Frequency band
            default: <unset> - all bands; example: bandname='K'
            <Options: '4'=48-96 MHz,'P'=298-345 MHz,'L'=1.15-1.75 GHz,
            'C'=4.2-5.1 GHz,'X'=6.8-9.6 GHz,'U'=13.5-16.3 GHz,
            'K'=20.8-25.8 GHz,'Q'=38-51 GHz>
    #projectname -- Observing project name
    #       default: <unset>; example='AP314'
    freqtol -- Tolerance in frequency shift in naming spectral windows
            default: channel width of current spectral window in Hz
            example: 150000.0

\end{verbatim}
\normalsize


%%%%%%%%%%%%%%%%%%%%%%%%%%%%%%%%%%%%%%%%%%%%%%%%%%%%%%%%%%%%%%%%%
\section{{\tt importasdm}}
\label{section:tasks.importasdm}

\small
\begin{verbatim}
    Convert an ALMA Science Data Model observation into a CASA visibility file (MS)
    
    Keyword arguments:
    asdm -- Name of input ASDM file (directory)
            default: <unset>; example: asdm='ExecBlock3'
\end{verbatim}
\normalsize


\section{{\tt importuvfits}}
\label{section:tasks.importuvfits}

\small
\begin{verbatim}
    Convert a UVFITS file to a CASA visibility data set (MS):
    
    Keyword arguments:
    fitsfile -- Name of input UV FITS file
            default: <unset>; example='3C273XC1.fits'
    vis -- Name of output visibility file (MS)
            default: <unset>; example: vis='3C273XC1.ms'
\end{verbatim}
\normalsize


%%%%%%%%%%%%%%%%%%%%%%%%%%%%%%%%%%%%%%%%%%%%%%%%%%%%%%%%%%%%%%%%%
\section{{\tt invert}}
\label{section:tasks.invert}

\small
\begin{verbatim}
    Calculate the dirty image and dirty beam:
    
    Keyword arguments:
    vis -- Name of input visibility file (MS)
            default: <unset>; example: vis='ngc5921.ms'
    imagename -- Name of output images: restored=imagename.restored;
            residual=imagename.residual,model=imagename
            default: <unset>; example: imagename='ngc5921'
    mode -- Type of selection 
            default: <unset>; example: mode='channel'; 
            <Options: 'none', channel'>
    nchan -- Number of channels to select
            default: 1; example: nchan=45
    start -- Start channel, 0-relative
            default=0; example: start=5
    width -- Channel width (value > 1 indicates channel averaging)
            default=1; example: width=5
    step -- Step in channel number
            default=1; example: step=2      
    imsize -- Image size in spatial pixels (x,y)
            default = [256,256]; example: imsize=[512,512]
    cell -- Cell size in arcseconds (x,y)
            default=[1,1]; example: cell=[0.5,0.5]
    stokes -- Stokes parameters to image
            default='I'; example: stokes='IQUV'; 
            <Options: 'I','IV','IQU','IQUV'>
    fieldid -- Field index identifier
            default=0; example: fieldid=1
    field -- Field name(s); this will use a minimum match on the strings
            that you provide; to use multiple strings, enter a single string
            with spaces separating the field names
            default: ''(will use fieldid); example: field='133',field='133 N5921'
    spwid -- Spectral window index identifier
            default=0; example: spwid=1
    weighting -- Weighting to apply to visibilities
            default='natural'; example: weighting='uniform'; 
            <Options: 'natural','uniform','briggs','radial', 'superuniform'>
    rmode -- Robustness mode (for use with Brigg's weighting)
            default='none'; example='abs'; 
            <Options: 'norm','abs','none'>
    robust -- Brigg's robustness parameter 
            default=0.0; example: robust=0.5; 
            <Options: -2.0 to 2.0; -2 (uniform)/+2 (natural)>
\end{verbatim}
\normalsize


%%%%%%%%%%%%%%%%%%%%%%%%%%%%%%%%%%%%%%%%%%%%%%%%%%%%%%%%%%%%%%%%%
\section{{\tt listhistory}}
\label{section:tasks.listhistory}

\small
\begin{verbatim}
    List the processing history of a dataset:
    
    Keyword arguments:
    vis -- Name of input visibility file (MS)
            default: <unset>; example: vis='ngc5921.ms'
\end{verbatim}
\normalsize


%%%%%%%%%%%%%%%%%%%%%%%%%%%%%%%%%%%%%%%%%%%%%%%%%%%%%%%%%%%%%%%%%
\section{{\tt listobs}}
\label{section:tasks.listobs}

\small
\begin{verbatim}
    List the observations in a dataset:
    
    Keyword arguments:
    vis -- Name of input visibility file (MS)
            default: <unset>; example: vis='ngc5921.ms'
    verbose -- List each observation in addition to the summary
            default=False; example: verbose=True
            <Options: True,False>
\end{verbatim}
\normalsize


%%%%%%%%%%%%%%%%%%%%%%%%%%%%%%%%%%%%%%%%%%%%%%%%%%%%%%%%%%%%%%%%%
\section{{\tt makemask}}
\label{section:tasks.makemask}

\small
\begin{verbatim}
    Derive/Calculate a mask image from another image or a visibility data set and 
       a set of imaging parameters:
    
    Keyword arguments:
    image -- Name of input image
            default: <unset>; example: image='ngc5921_task.image'
            <Note: If this parameter is used, interactive is assume
            parameters from vis and below are not needed/used>
    interactive -- Indicate whether you want to view the image and
            interactively select regions.
            default: False; example: interactive=True
            <Note: This isn't currently enabled>
    cleanbox -- List of [blc-x,blc-y,trc-x,trc-y] values
            default: []; example: cleanbox=[110,110,150,145]
            Note: This can also be a filename with clean values:
            fieldindex blc-x blc-y trc-x trc-y
    expr -- An expression specifying the mask
            default: ''; example: expr='somestring'
            <Note: This isn't currently enabled>
    ----------------------------------------------------------------
    vis -- Name of input visibility file (MS)
            default: <unset>; example: vis='ngc5921.ms'
    imagename -- Name of output mask image
            default: <unset>; example: imagename='ngc5921.mask'
    mode -- Type of selection 
            default: <unset>; example: mode='channel'; 
            <Options: 'none', channel'>
    nchan -- Number of channels to select
            default: 1; example: nchan=45
    start -- Start channel, 0-relative
            default=0; example: start=5
    width -- Channel width (value > 1 indicates channel averaging)
            default=1; example: width=5
    step -- Step in channel number
            default=1; example: step=2      
    imsize -- Image size in spatial pixels (x,y)
            default = [256,256]; example: imsize=[512,512]
    cell -- Cell size in arcseconds (x,y)
            default=[1,1]; example: cell=[0.5,0.5]
    stokes -- Stokes parameters to image
            default='I'; example: stokes='IQUV'; 
            <Options: 'I','IV','IQU','IQUV'>
    fieldid -- Field index identifier
            default=0; example: fieldid=1
    field -- Field name(s); this will use a minimum match on the strings
            that you provide; to use multiple strings, enter a single string
            with spaces separating the field names
            default: ''(will use fieldid); example: field='133',field='133 N5921'
    spwid -- Spectral window index identifier
            default=0; example: spwid=1
\end{verbatim}
\normalsize


%%%%%%%%%%%%%%%%%%%%%%%%%%%%%%%%%%%%%%%%%%%%%%%%%%%%%%%%%%%%%%%%%
\section{{\tt mosaic}}
\label{section:tasks.mosaic}

\small
\begin{verbatim}
    Calculate a multi-field deconvolved image with selected clean algorithm:
    
    Keyword arguments:
    vis -- Name of input visibility file (MS)
            default: <unset>; example: vis='ngc5921.ms'
    imagename -- Name of output images: restored=imagename.restored;
            residual=imagename.residual,model=imagename
            default: <unset>; example: imagename='ngc5921'
    mode -- Type of selection 
            default: False; example: mode='channel'; 
            <Options: 'none', channel'>
    mfalg -- Algorithm to use
            default: 'mfclark'; example: mfalg='mfhogbom'; 
            <Options: 'mfclark','mfhogbom','mfmultiscale'>
    niter -- Number iterations, set to zero for no CLEANing
            default: 500; example: niter=500
    gain -- Loop gain for CLEANing
            default: 0.1; example: gain=0.1
    threshold -- Flux level at which to stop CLEANing (units=mJy)
            default: 0.0; example: threshold=0.0
    mask -- Name(s) of mask image(s) used for CLEANing
            default: <unset>; example: mask='orion.mask'
    nchan -- Number of channels to select
            default: 1; example: nchan=45
    start -- Start channel, 0-relative
            default=0; example: start=5
    width -- Channel width (value > 1 indicates channel averaging)
            default=1; example: width=5
    step -- Step in channel number
            default=1; example: step=2      
    imsize -- Image size in spatial pixels (x,y)
            default = [256,256]; example: imsize=[512,512]
    cell -- Cell size in arcseconds (x,y)
            default=[1,1]; example: cell=[0.5,0.5]
    stokes -- Stokes parameters to image
            default='I'; example: stokes='IQUV'; 
            <Options: 'I','IV','IQU','IQUV'>
    fieldid -- Field index identifier
            default=0; example: fieldid=1
    field -- Field name(s); this will use a minimum match on the strings
            that you provide; to use multiple strings, enter a single string
            with spaces separating the field names
            default: ''(will use fieldid); example: field='133',field='133 N5921'
    phasecenter -- Phase center (field identifier or direction string)
            default=0; example: phasecenter=62
            <Note: direction strings aren't currently handled>
    spwid -- Spectral window index identifier
            default=0; example: spwid=1
    weighting -- Weighting to apply to visibilities
            default='natural'; example: weighting='uniform'; 
            <Options: 'natural','uniform','briggs','radial', 'superuniform'>
    rmode -- Robustness mode (for use with Brigg's weighting)
            default='none'; example='abs'; 
            <Options: 'norm','abs','none'>
    robust -- Brigg's robustness parameter 
            default=0.0; example: robust=0.5; 
            <Options: -2.0 to 2.0; -2 (uniform)/+2 (natural)>
    scaletype -- Image plane flux scale type
            default='NONE'; example: scaletype='SAULT'
            <Options: "NONE","SAULT">
    constpb -- In Sault weighting, the flux scale is constant above this PB level
            default=0.4; example: constpb=0.3
    minpb -- Minimum PB level to use
            default=0.1; example: minpb=0.15
\end{verbatim}
\normalsize


%%%%%%%%%%%%%%%%%%%%%%%%%%%%%%%%%%%%%%%%%%%%%%%%%%%%%%%%%%%%%%%%%
\section{{\tt plotants}}
\label{section:tasks.plotants}

\small
\begin{verbatim}
    Plot the antenna distribution in local reference frame; 
    X-toward local east; Y-toward local north:
    
    Keyword arguments:
    vis -- Name of input visibility file (MS)
            default: <unset>; example: vis='ngc5921.ms'

\end{verbatim}
\normalsize


%%%%%%%%%%%%%%%%%%%%%%%%%%%%%%%%%%%%%%%%%%%%%%%%%%%%%%%%%%%%%%%%%
\section{{\tt plotcal}}
\label{section:tasks.plotcal}

\small
\begin{verbatim}
        Plot calibration solutions: 
        
        Keyword arguments:
        tablein -- Name of input calibration table 
                default: <unset>; example: vis='ngc5921.gcal'
        yaxis -- Visibility data to plot along the y-axis
                <Options: 'amp','phase','rlphase','xyphase','delay','delayrate'>
        antennaid -- Antenna index identifier(s)
                default: -1 (all); example: antennaid=[13]
        caldescid -- Calibrater data description ID (combination of SPW and polarization)
                default: -1 (all); example: caldescid=0
        --- Plot Options ---
        nxpanel -- Panel number in the x-direction
                default: 1; example: nxpanel=2
        nypanel -- Panel number in the y-direction 
                default: 1; example: nypanel=2
        multiplot -- Plot data (e.g., from different antennas) on separate plots
                default: False; example: multiplot=True
\end{verbatim}
\normalsize


%%%%%%%%%%%%%%%%%%%%%%%%%%%%%%%%%%%%%%%%%%%%%%%%%%%%%%%%%%%%%%%%%
\section{{\tt plotxy}}
\label{section:tasks.plotxy}

\small
\begin{verbatim}
    Plot points for selected X and Y axes:
    
    Keyword arguments:
    vis -- Name of input visibility file (MS)
            default: <unset>; example: vis='ngc5921.ms'
    xaxis -- Visibility file (MS) data to plot along the x-axis
            <Options: 'azimuth','baseline','channel','elevation',
            'hourangle','parallacticangle','time','u','uvdist','x'>
    yaxis -- Visibility data to plot along the y-axis
            <Options: 'amp','phase','v'>
    datacolumn -- Visibility file (MS) data column.
            default: 'corrected'; example: datacolumn='model'
            <Options: 'data' (raw),'corrected','model','weight'>
    antennaid -- Antenna index identifier
            default: -1 (all); example: antennaid=[13]
    spwid -- Spectral window index identifier
            default: -1 (all); example: spwid=0
    fieldid -- Field index identifier
            default: -1 (all); example: fieldid=0
    field -- Field name(s); this will use a minimum match on the strings
            that you provide; to use multiple strings, enter a single string
            with spaces separating the field names
            default: ''(will use fieldid); example: field='133',field='133 N5921'
    timerange
    correlations -- Correlations to plot
            default: '' (all); example: correlations='RR'
            <Options: '','RR','LL','RR LL','XX','YY','XX YY'>
    --- Spectral Information ---
    nchan -- Number of channels to select
            default: -1 (all); example: nchan=55
    start -- Start channel (0-relative)
            default=0; example: start=5
    width -- Channel width (value>0 indicates averaging)
            default=1; example: width=5
    --- Plot Options ---
    subplot -- Panel number on the display screen
            default: 111 (full screen display); example:
               subplot=221 (upper left panel of 4 on frame)
               subplot=223 (lower left panel of 4 on frame)
               subplot=212 (lower panel of 2 on frame)
            <Options: subplot=yxn; where y=number of rows,
            x=number of columns, n=panel number (numbered starting
            at 1 from top left)>
    overplot -- Overplot these values on current plot (if possible)
            default: False; example: overplot=True
    plotsymbol -- pylab plot symbol
            default: ','; example: plotsymbol='bo' (blue circles)
            <Options: '-' = solid line
                      'o' = filled circles
                      '.' = points
                      '-.ro' = combined dash dot line with red circles
    title -- Plot title (above plot)
            default: ''; example: title='This is my title'
    xlabels -- Label for x axis
            default: ''; example: xlabels='X Axis'
    ylabels -- Label for y axis
            default: ''; example: ylabels='Y Axis'
    iteration -- Plot each value of this data axis on a separate plot
            default: '' (no iteration); example: iteration='antenna'
            <Options: 'antenna1','field_id','baseline','time'>
    fontsize -- Font size for labels
            default: 1; example: fontsize=2
    windowsize -- Window size
            default: 1.0; example: windowsize=0.5
\end{verbatim}
\normalsize


%%%%%%%%%%%%%%%%%%%%%%%%%%%%%%%%%%%%%%%%%%%%%%%%%%%%%%%%%%%%%%%%%
\section{{\tt pointcal}}
\label{section:tasks.pointcal}

\small
\begin{verbatim}
    Solve for pointing error calibration:
    
    Keyword arguments:
    vis -- Name of input visibility file (MS)
            default: <unset>; example: vis='ngc5921.ms'
    model -- Name of input model (component list or image)
            default: <unset>; example: model='ngc5921.im'
    caltable -- Name of output Pointing calibration table
            default: <unset>; example: caltable='ngc5921.gcal'
    mode -- Type of data selection
            default: 'none' (all data); example: mode='channel'
            <Options: 'channel', 'velocity', 'none'
    nchan -- Number of channels to select (when mode='channel')
            default: 1; example: nchan=45 (note: nchan=0 selects all)
    start -- Start channel, 0-relative;active when mode='channel','velocity'
            default=0; example: start=5,start='20km/s'
    step -- Increment between channels;active when mode='channel','velocity'
            default=1; example: step=1, step='1km/s'
    msselect -- Optional subset of the data to select (field,spw,time,etc)
            default:'';example:msselect='FIELD_ID==0', 
            msselect='FIELD_ID IN [0,1]', msselect='FIELD_ID <= 1'
            <Options: see: http://aips2.nrao.edu/docs/notes/199/199.html>
    gaincurve -- Apply VLA antenna gain curve correction
            default: True; example: gaincurve=False
            <Options: True, False>
    opacity -- Apply opacity correction
            default: True; example: opacity=False
            <Options: True, False>
    tau -- Opacity value to apply (if opacity=True)
            default:0.0001; example: tau=0.0005
    solint --  Solution interval (sec)
            default: -1 (scan based); example: solint=60.
    refant -- Reference antenna *name*
            default: -1 (any integer=none); example: refant='13'
\end{verbatim}
\normalsize


%%%%%%%%%%%%%%%%%%%%%%%%%%%%%%%%%%%%%%%%%%%%%%%%%%%%%%%%%%%%%%%%%
\section{{\tt smooth}}
\label{section:tasks.smooth}

\small
\begin{verbatim}

        Smooth calibration solution(s) derived from one or more sources:
        
        Keyword arguments:
        vis -- Name of input visibility file (MS)
                default: <unset>; example: vis='ngc5921.ms'
        tablein -- Input calibration table (any type)
                default: <unset>; example: tablein='ngc5921.gcal'
        caltable -- Output calibration table (smoothed)
                default: <unset>; example: caltable='ngc5921_smooth.gcal'
        append -- Append solutions to an existing calibration table?
                default: False; example: append=True
        calselect -- Optional subset of calibration data to select (e.g., field)
                default: ''; example: calselect='FIELD_NAME=FIELD1'
        smoothtype -- The smoothing filter to be used
                default: 'mean'; example: smoothtype='median'
                <Options: 'mean','median','smean','smedian','none'=copy>
        smoothtime -- Smoothing filter time (sec)
                default: 0.0; example: smoothtime=10.
\end{verbatim}
\normalsize


%%%%%%%%%%%%%%%%%%%%%%%%%%%%%%%%%%%%%%%%%%%%%%%%%%%%%%%%%%%%%%%%%
\section{{\tt setjy}}
\label{section:tasks.setjy}

\small
\begin{verbatim}
    Compute the model visibility for a specified source flux density: 
    
    Keyword arguments:
    vis -- Name of input visibility file (MS)
            default: <unset>; example: vis='ngc5921.ms'
    fieldid -- Field index identifier (0-based)
            default=-1 (all); example: fieldid=1
    field -- Field name(s); this will use a minimum match on the strings
            that you provide; to use multiple strings, enter a single string
            with spaces separating the field names
            default: ''(will use fieldid); example: field='133',field='133 N5921'
    spwid -- Spectral window index identifier (0-based)
            default=-1 (all); example: spwid=1
    fluxdensity -- Specified flux density (I,Q,U,V) in Jy
            default=-1 (lookup the value; use 1.0 if not found)
            example: fluxdensity=[2.6,0.2,0.3,0.5]
    standard -- Flux density standard
            default: 'Perley-Taylor 99'; example: standard='Baars'
            <Options: 'Baars','Perley 90','Perley-Taylor 95',
            'Perley-Taylor 99'>
\end{verbatim}
\normalsize


%%%%%%%%%%%%%%%%%%%%%%%%%%%%%%%%%%%%%%%%%%%%%%%%%%%%%%%%%%%%%%%%%
\section{{\tt split}}
\label{section:tasks.split}

\small
\begin{verbatim}
    Create a new data set (MS) from a subset of an existing data set (MS):
    
    Keyword arguments:
    vis -- Name of input visibility file (MS)
            default: <unset>; example: vis='ngc5921.ms'
    outputvis -- Name of output visibility file (MS)
            default: <unset>; example: outputvis='ngc5921_src.ms'
    fieldid -- Field index identifier
            default=-1 (all); example: fieldid=1
    field -- Field name(s); this will use a minimum match on the strings
            that you provide; to use multiple strings, enter a single string
            with spaces separating the field names
            default: ''(will use fieldid); example: field='133',field='133 N5921'
    spwid -- Spectral window index identifier
            default=-1 (all); example: spwid=1
    nchan -- Number of channels to select
            default:-1 (all); example: nchan=45
    start -- Start channel, 0-relative
            default=0; example: start=5
    step -- Step in channel number
            default=1; example: step=2      
    timebin -- Value for time averaging
            default='-1s' (no averaging); example: timebin='30s'
    timerange -- Select time range subset of data
            default=''; 
            example: timerange='YYYY/MM/DD/HH:MM:SS.sss'
            timerange='< YYYY/MM/DD/HH:MM:SS.sss'
            timerange='> YYYY/MM/DD/HH:MM:SS.sss'
            timerange='ddd/HH:MM:SS.sss'
            timerange='< ddd/HH:MM:SS.sss'
            timerange='> ddd/HH:MM:SS.sss'
    datacolumn -- Which data set column to split out
            default='corrected'; example: datacolumn='data'
            <Options: 'data', 'corrected', 'model'>
\end{verbatim}
\normalsize


%%%%%%%%%%%%%%%%%%%%%%%%%%%%%%%%%%%%%%%%%%%%%%%%%%%%%%%%%%%%%%%%%
\section{{\tt uvmodelfit}}
\label{section:tasks.uvmodelfit}

\small
\begin{verbatim}
    Fit a single component source model to the uv data:
    
    Keyword arguments:
    vis -- Name of input visibility file (MS)
            default: <unset>; example: vis='ngc5921.ms'
    niter -- Number of fitting iterations to execute
            default: 0; example: niter=5
    comptype -- component model type
            default: 'P'; example: comptype='G'
            <Options: 'P' (point source), 'G' (gaussian), 'D' (elliptical disk)>
    sourcepar -- Starting guess for component parameters (flux,xoffset,yoffset)
            default: [1,0,0]; example: sourcepar=[2.5,0.3,0.1]
            Note: Flux is in Jy, xoffset is in arcsec, yoffset is in arcsec.
    fixpar -- Control which parameters to let vary in the fit
            default: [] (all vary); example: vary=[False,True,True]
            (this would fix the flux to that set in sourcepar but allow the
            x and y offset positions to be fit).
    file -- Optional output component list table
            default: ''; example: file='componentlist.cl'
    --- Data Selection ---
    mode -- Type of data selection
            default: 'none' (all data); example: mode='channel'
            <Options: 'channel', 'velocity', 'none'
    nchan -- Number of channels to select (when mode='channel')
            default: 1; example: nchan=45 (note: nchan=0 selects all)
    start -- Start channel, 0-relative;active when mode='channel','velocity'
            default=0; example: start=5,start='20km/s'
    step -- Increment between channels;active when mode='channel','velocity'
            default=1; example: step=1, step='1km/s'
    msselect -- Optional subset of the data to select (field,spw,time,etc)
            default:'';example:msselect='FIELD_ID==0', 
            msselect='FIELD_ID IN [0,1]', msselect='FIELD_ID <= 1'
            <Options: see: http://aips2.nrao.edu/docs/notes/199/199.html>
\end{verbatim}
\normalsize


%%%%%%%%%%%%%%%%%%%%%%%%%%%%%%%%%%%%%%%%%%%%%%%%%%%%%%%%%%%%%%%%%
\section{{\tt viewer}}
\label{section:tasks.viewer}

\small
\begin{verbatim}

    View an image or visibility data set:
    
    Keyword arguments:
    imagename -- Name of file to visualize
            default: <unset>; example: imagename='ngc5921.image'
\end{verbatim}
\normalsize

%%%%%%%%%%%%%%%%%%%%%%%%%%%%%%%%%%%%%%%%%%%%%%%%%%%%%%%%%%%%%%%%%
%%%%%%%%%%%%%%%%%%%%%%%%%%%%%%%%%%%%%%%%%%%%%%%%%%%%%%%%%%%%%%%%%

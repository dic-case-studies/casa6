%%%%%%%%%%%%%%%%%%%%%%%%%%%%%%%%%%%%%%%%%%%%%%%%%%%%%%%%%%%%%%%%%
%%%%%%%%%%%%%%%%%%%%%%%%%%%%%%%%%%%%%%%%%%%%%%%%%%%%%%%%%%%%%%%%%
%%%%%%%%%%%%%%%%%%%%%%%%%%%%%%%%%%%%%%%%%%%%%%%%%%%%%%%%%%%%%%%%%

% STM 2007-04-13  split from previous version
% STM 2007-10-10  update to tasks in beta
% JO  2011-10-05  Release 3.3 updates
% JO 2014-02-01 typos

\chapter[Appendix: CASA Dictionaries]{CASA Dictionaries}
\label{chapter:dict}

{BETA ALERT:} These tend to become out of date as we add new tasks
or change names.

\section{AIPS -- CASA dictionary}
\label{section:dict.aips}

In Table\,\ref{table:aips} we provide a comparison of CASA and AIPS
commands. The data reduction recipes and break-down of jobs in
individual tasks, however, is not the same in both
packages. Nevertheless, the table may give AIPS users a good start if
they search for functionality in CASA. 


%\vspace{5mm}

\begin{table}[ht]
\footnotesize
\caption[AIPS -- CASA dictionary]
        {\label{table:aips}AIPS -- CASA dictionary}
\begin{center}
\begin{tabular}{|c|c|c|} \hline
{\bf AIPS Task} &  {\bf CASA task/tool}     &  {\bf Description}   \\
APROPOS  & taskhelp 	 & List tasks with a short description of their purposes\\
BLCAL 	 & blcal   	 & Calculate a baseline-based gain calibration solution\\
BLCHN 	 & blcal   	 & Calculate a baseline-based bandpass calibration solution\\
BPASS 	 & bandpass 	 & Calibrate bandpasses\\
CALIB 	 & gaincal 	 & Calibrate gains (amplitudes and phases)\\
CLCAL 	 & applycal 	 & Apply calibration to data\\
COMB 	 & immath  	 & Combine images\\
CPASS 	 & bandpass   	 & Calibrate bandpasses by polynomial fitting\\
DBCON 	 & concat 	         & Concatenate u-v datasets\\
DEFAULT	 & default 	 & Load a task with default parameters\\
FILLM 	 & importvla 	 & Import old-format VLA data\\
FITLD 	 & importuvfits 	 & Import a u-v dataset which is in FITS format\\
FITLD 	 & importfits 	 & Import an image which is in FITS format\\
FITTP 	 & exportuvfits 	 & Write a u-v dataset to FITS format\\
FITTP 	 & exportfits 	 & Write an image to FITS format\\
FRING 	 & ---     	 & Calibrate group delays and phase rates.\\
GETJY 	 & fluxscale 	 & Determine flux densities for other cals\\
GO 	 & go      	 & Run a task\\
HELP 	 & help 	         & Display the help page for a task\\
IMAGR 	 & clean 	         & Image and deconvolve\\
IMFIT 	 & imfit    	 & Fit gaussian components to an image\\
IMHEAD 	 & vishead 	 & View header for u-v data\\
IMHEAD 	 & imhead 	         & View header for an image\\
IMLIN 	 & imcontsub 	 & Subtract continuum in image plane\\
IMLOD 	 & importfits 	 & Import a FITS image\\
IMSTAT 	 & imstat   	 & Measure statistics on an image\\
INP 	 & inp     	 & View task parameters\\
JMFIT 	 & imfit    	 & Fit gaussian components to an image\\
LISTR 	 & listobs 	 & Print basic data\\
MCAT 	 & ls 	         & List image data files\\
MOMNT 	 & immoments 	 & Compute moments from an image\\
OHGEO 	 & imregrid 	 & Regrids an image onto another image's geometry\\
PBCOR 	 & immath   	 & Correct an image for the primary beam\\
PCAL 	 & polcal 	         & Calibrate polarization\\
POSSM 	 & plotcal 	 & Plot bandpass calibration tables\\
POSSM 	 & plotms    	 & Plot spectra\\
PRTAN 	 & listobs 	 & Print antenna locations\\
PRTAN 	 & plotants 	 & Plot antenna locations\\
QUACK 	 & flagdata 	 & Remove first integrations from scans\\
RENAME 	 & mv 	         & Rename an image or dataset\\
SETJY 	 & setjy 	         & Set flux densities for flux cals\\
SMOTH 	 & imsmooth 	 & Smooth an image\\
SNPLT 	 & plotcal 	 & Plot gain calibration tables\\
SPFLG 	 & viewer   	 & Flag raster image of time v. channel\\
SPLIT 	 & split    	 & Write out u-v files for individual sources\\
TASK 	 & inp     	 & Load a task with current parameters\\
TGET 	 & tget    	 & Load a task with parameters last used for that task\\
TVALL 	 & viewer  	 & Display image\\
TVFLG 	 & viewer 	         & Flag raster image of time v. baseline\\
UCAT 	 & ls      	 & List u-v data files\\
UVFIX 	 & fixvis  	 & Compute u, v, and w coordinates\\
UVFLG 	 & flagdata 	 & Flag data\\
UVLIN 	 & uvcontsub 	 & Subtract continuum from u-v data\\
UVLSF 	 & uvcontsub 	 & Subtract continuum from u-v data\\
UVPLT 	 & plotms  	 & Plot u-v data\\
UVSUB 	 & uvsub    	 & Subtracts model u-v data from corrected u-v data\\
WIPER 	 & plotms 	         & Plot and flag u-v data\\
ZAP 	 & rmtables 	 & Delete data files \\
\hline
\end{tabular}
\end{center}
\end{table}
\normalsize

%%%%%%%%%%%%%%%%%%%%%%%%%%%%%%%%%%%%%%%%%%%%%%%%%%%%%%%%%%%%%%%%%
\section{MIRIAD -- CASA dictionary}
\label{section:dict.miriad}

Table~\ref{table:miriad} provides a list of common Miriad tasks, and their
equivalent CASA tool or tool function names. The two packages differ
in both their architecture and calibration and imaging models, and
there is often not a direct correspondence. However, this index does
provide a scientific user of CASA who is familiar with MIRIAD, with
a simple translation table to map their existing data reduction
knowledge to the new package.

%\vspace{5mm}

\begin{table}[ht] \footnotesize
\caption[MIRIAD -- CASA dictionary]
        {\label{table:miriad}MIRIAD -- CASA dictionary}
\begin{center}
\begin{tabular}{|c|c|c|} \hline
{\bf MIRIAD Task} &  {\bf Description}     &  {\bf CASA task/tool}   \\
%  atlod   &  load ATCA data                &  atcafiller tool \\
  blflag  &  Interactive baseline based editor/flagger  &  mp raster displays\\
  cgcurs  &  Interactive image analysis    &  viewer \\ 
  cgdisp  &  Image display, overlays       &  viewer  \\
  clean   &  Clean an image                &  clean   \\
  fits    &  FITS image filler             &  importfits, exportfits,
  importuvfits, exportuvfits \\  
  gpboot  &  Set flux density scale        &  fluxscale  \\
  gpcal   &  Polarization leakage and gain calibration  &  gaincal  \\
  gpcopy  &  copy calibration tables       &  {\it not needed} \\
  gpplt   &  Plot calibration solutions    &  plotcal  \\
  imcomb  &  Image combination             &  immaths   \\
  imfit   &  Image-plane component fitter  & imfit  \\
  impol   &  Create polarization images    &  clean   \\
  imstat  &  Image statistics              &  imstats  \\
  imsub   &  Extract sub-image             &  ia.subimage   \\
  invert  &  Synthesis imaging             &  clean  \\
  linmos  &  linear mosaic combination of images  &  clean  \\
  maths   &  Calculations involving images  &   immath \\ 
  mfcal   &  Bandpass and gain calibration  &  bandpass \\
  prthd   &  Print header of image or uvdata  &  imhead, listobs, vishead  \\
  restor  &  Restore a clean component model  &  clean \\
  selfcal &  selfcalibration of visibility data  & clean, gaincal, etc.\\ 
\hline
\end{tabular}
\end{center}
\end{table}
\normalsize

%%%%%%%%%%%%%%%%%%%%%%%%%%%%%%%%%%%%%%%%%%%%%%%%%%%%%%%%%%%%%%%%%

\section{CLIC -- CASA dictionary}
\label{section:dict.clic}

Table~\ref{table:clic} provides a list of common CLIC tasks, and their
equivalent CASA tool or tool function names. The two packages are
very similar since the CASA software to reduce IRAM data is based
on the CLIC reduction procedures.

\vspace{5mm}
\begin{table}[hb]
\caption[CLIC--CASA dictionary]
        {\label{table:clic} CLIC--CASA dictionary}
\begin{center}
\begin{tabular}{|c|c|c|} \hline
{\bf CLIC Function}  & {\bf Description}           &   {\bf CASA task/tool}  \\
  load              & Load data                    &  importfits,
  importasdm, importuvfits \\
  print             & Print text summary of data   &  listobs \\ 
  flag              & Flag data                    &  plotms, flagdata, viewer \\
  phcor             & Atmospheric phase correction &  gaincal \\  
  rf                & Radio frequency bandpass     &  bandpass \\
  phase             & Phase calibration            &  gaincal \\
  flux              & Absolute flux calibration    &  setjy, fluxscale \\
  ampl              & Amplitude calibration        &  gaincal \\
  table             & Split out calibrated data (uv table)  &   split \\
\hline
\end{tabular}
\end{center}
\end{table}

%%%%%%%%%%%%%%%%%%%%%%%%%%%%%%%%%%%%%%%%%%%%%%%%%%%%%%%%%%%%%%%%%
%%%%%%%%%%%%%%%%%%%%%%%%%%%%%%%%%%%%%%%%%%%%%%%%%%%%%%%%%%%%%%%%%

% STM 2007-04-13  split from previous version
% STM 2007-10-11  add into beta
% STM 2007-10-22  put in appendix for Beta Release
% RI  2009-12-17  Release 0 (3.0.0) draft
% STM 2009-12-21  Release 0 (3.0.0) final
% RI 2010-04-10 Release 3.0.1
% JO 2013-12-17 Release 4.2 edits from Jen

\chapter[Simulation]{Simulation}
\label{chapter:sim}

{\bfseries New in 4.2:}

\noindent $\bullet$ Single dish imaging in CASA is in a state of vigorous development.  At
the time of the 4.2 release, single dish imaging is accomplished in
{\tt simalma} and {\tt simobserve} by calling the {\tt sdimaging} task
with spheroidal gridding, and particular choices of cell size and
convolution support.  These parameters may change as ALMA best
practice is refined.  CAVEAT: There is currently a 15\% uncertainty in the
beam area and resulting absolute flux scale for simulated single dish
images.

  \noindent $\bullet$ The {\tt simalma} task has been significantly
  improved in 4.2 - in particular, the {\tt dryrun} parameter allows
  users to test their input parameters and skymodel, and see a report
  of what {\tt simalma} will do without waiting.  The report
  includes information on the multiple requested components (12m
  interferometric array, 7m interferometric array, and total power)
  and whether any input parameters are likely to cause issues in
  simulation.  The task's parameters have been changed to more easily
  support multiple configurations and components in a single run.
  CAVEAT: The tools do not split up long observations at this time, so
  a 12h or longer observation will include time at low elevation,
  which may be unrealistic.  The user may need to manually split up
  their simulation, and use multiple calls to {\tt simobserve} with
  shorter totaltime.

\noindent $\bullet$ {\tt simobserve} adds noise to an observation by
default - in previous CASA versions the default was noiseless.

\noindent $\bullet$ {\tt simanalyze} now defaults to producing a dirty
image (\tt niter}=0.


%% {\bfseries Important note on task names:}
%% Users are encouraged to use {\tt simobserve} and {\tt simanalyze}.  The combined task {\tt simdata} is still present, but will be removed in the future.

The tasks available for simulating observations are:
\begin{itemize}
\item {\tt simobserve} --- simulate an interferometer or total power
observation (\S~\ref{section:sim.almasimmos})
\item {\tt simanalyze} --- image and analyze simulated data sets
(\S~\ref{section:sim.almasimmos})
\item {\tt simalma} --- simulate an ALMA observation including
  multiple configurations of the 12-m interferometric array, the 7-m
  ACA, and total power measurements. Generate a combined image from
  the simulated data sets (\S~\ref{section:sim.simalma})
\end{itemize}

The capability of simulating observations and data sets from the JVLA
and ALMA are an important use-case for CASA.  This not only allows one
to get an idea of the capabilities of these instruments for doing
science, but also provides benchmarks for the performance and utility
of the software to process ``realistic'' data sets (with atmospheric
and instrumental effects).  Simulations can also be used to tune
parameters of the data reduction and therefore help to optimize the
process.  CASA can calculate visibilities (create a measurement set)
for any interferometric array, and calculate and apply calibration
tables representing some of the most important corrupting
effects. {\tt simobserve} can also simulate total power observations,
which can be combined with interferometric data in {\tt simanalyze}
(i.e. one would run {\tt simobserve} twice, {\tt simanalyze} once).
The task {\tt simalma} is a task to simulate an ALMA observation,
including ALMA 12-m, ACA 7-m and total power arrays, and generate a
combined image.  {\tt simalma} also attempts to provide useful
feedback on those different observation components, to help the user
better understand the observing considerations.


\begin{wrapfigure}{r}{2.5in}
 \begin{boxedminipage}{2.5in}
    \centerline{\bf Inside the Toolkit:}
    The simulator methods are in the {\tt sm} tool.
    Many of the other tools are also helpful when
    constructing and analyzing simulations.
 \end{boxedminipage}
\end{wrapfigure}

CASA's simulation capabilities continue to be improved with each CASA release.
For the most current information, please refer to
\url{http://www.casaguides.nrao.edu}, and click on
``Simulating Observations in CASA''.
%
Following general CASA practice, the greatest flexibility and richest
functionality is at the Toolkit level.  The most commonly used
procedures for interferometric and single dish simulation are
encapsulated in the {\tt simobserve} task.


%%%%%%%%%%%%%%%%%%%%%%%%%%%%%%%%%%%%%%%%%%%%%%%%%%%%%%%%%%%%%%%%%
%%%%%%%%%%%%%%%%%%%%%%%%%%%%%%%%%%%%%%%%%%%%%%%%%%%%%%%%%%%%%%%%%
\section{Simulating ALMA observations with {\tt simobserve} and {\tt simanalyze}}
\label{section:sim.almasimmos}

The {\tt simobserve} inputs are (submenus expand slightly differently for thermalnoise=manual and single dish observing):
\small
\begin{verbatim}
project             =      'sim'        #  root prefix for output file names
skymodel            =         ''        #  model image to observe
     inbright       =         ''        #  scale surface brightness of
                                        #  brightest pixel e.g. "1.2Jy/pixel"
     indirection    =         ''        #  set new direction
                                        #   e.g. "J2000 19h00m00 -40d00m00"
     incell         =         ''        #  set new cell/pixel size
                                        #   e.g. "0.1arcsec"
     incenter       =         ''        #  set new frequency of center
                                        #  channel e.g. "89GHz"
                                        #   (required even for 2D model)
     inwidth        =         ''        #  set new channel width
                                        #  e.g. "10MHz" (required even
                                        #   for 2D model)

complist            =         ''        #  componentlist to observe
     compwidth      =     '8GHz'        #  bandwidth of components

setpointings        =       True        
     integration    =      '10s'        #  integration (sampling) time
     direction      =         ''        #  "J2000 19h00m00 -40d00m00"
                                        #   or "" to center on model
     mapsize        =   ['', '']        #  angular size of map or ""
                                        #  to cover model
     maptype        =     'ALMA'        #  hexagonal, square (raster),
                                        #  ALMA, etc.
     pointingspacing =         ''       #  spacing in between
                                        #  pointings or "0.25PB" or ""
                                        #  for ALMA default
                                        #   INT=lambda/D/sqrt(3), SD=lambda/D/3

obsmode             =      'int'        #  observation mode to simulate
                                        #  [int(interferometer)|sd(singledish)|""(none)]
     antennalist    = 'alma.out10.cfg'  #  interferometer antenna position file
     refdate        = '2014/05/21'      #  date of observation - not
                                        #   critical unless concatenating
                                        #   simulations
     hourangle      =  'transit'        #  hour angle of observation
                                        #  center e.g. "-3:00:00", "5h", "-4.5" (a
                                        #   number without units will
                                        #   be interpreted as hours), or "transit"
     totaltime      =    '7200s'        #  total time of observation
                                        #   or number of repetitions
     caldirection   =         ''        #  pt source calibrator [experimental]
     calflux        =      '1Jy'        

thermalnoise        = 'tsys-atm'        #  add thermal noise: 
                                        #  [tsys-atm|tsys-manual|""]
     user_pwv       =        1.0        #  Precipitable Water Vapor in mm
     t_ground       =      269.0        #  ambient temperature
     seed           =      11111        #  random number seed

leakage             =        0.0        #  cross polarization (interferometer only)
graphics            =     'both'        #  display graphics at each
                                        #   stage to [screen|file|both|none]
verbose             =      False        
overwrite           =       True        #  overwrite files starting with $project
async               =      False        #  If true the taskname must
                                        #   be started using simobserve(...)
\end{verbatim}
\normalsize

This task takes an input model image or list of components, plus a
list of antennas (locations and sizes), and simulates a particular
observation (specified by mosaic setup and observing cycles and
times).  The output is a measurement set suitable for further analysis in CASA.

The {\tt simanalyze} inputs are:
\small
\begin{verbatim}
project             =      'sim'        #  root prefix for output file names
image               =       True        #  (re)image $project.*.ms to $project.image
     vis            =  'default'        #  Measurement Set(s) to image
     modelimage     =         ''        #  lower resolution prior
                                        #  image to use in clean e.g. existing total
                                        #   power image
     imsize         =          0        #  output image size in pixels
                                        #   (x,y) or 0 to match model
     imdirection    =         ''        #  set output image direction,
                                        #   (otherwise center on the model)
     cell           =         ''        #  cell size with units or "" to equal model
     interactive    =      False        #  interactive clean?  (make
                                        #   sure to set niter>0 also)
     niter          =          0        #  maximum number of
                                        #  iterations (0 for dirty image)
     threshold      =   '0.1mJy'        #  flux level (+units) to stop cleaning
     weighting      =  'natural'        #  weighting to apply to
                                        #  visibilities briggs will use robust=0.5
     mask           =         []        #  Cleanbox(es), mask
                                        #  image(s), region(s), or a level
     outertaper     =         []        #  uv-taper on outer baselines in uv-plane
     stokes         =        'I'        #  Stokes params to image
     featherimage   =         ''        #  image (e.g. total power) to
                                        #  feather with new image

analyze             =       True        #  (only first 6 selected
                                        #   outputs will be displayed)
     showuv         =       True        #  display uv coverage
     showpsf        =       True        #  display synthesized (dirty)
                                        #  beam (ignored in single dish simulation)
     showmodel      =       True        #  display sky model at original resolution
     showconvolved  =      False        #  display sky model convolved with output beam
     showclean      =       True        #  display the synthesized image
     showresidual   =      False        #  display the clean residual
                                        #  image (ignored in single dish simulation)
     showdifference =       True        #  display difference between
                                        #  output cleaned image and input
                                        #   model sky image convolved
                                        #   with output clean beam
     showfidelity   =       True        #  display fidelity

graphics            =     'both'        #  display graphics at each
                                        #  stage to [screen|file|both|none]
verbose             =      False        
overwrite           =       True        #  overwrite files starting with $project
async               =      False        #  If true the taskname must
                                        #  be started using simanalyze(...)
\end{verbatim}
\normalsize

This task analyzes one or more measurement sets - interferometric and/or single dish.
The output is a synthesized image created from those visibilities, a difference image
between the synthesized image and your sky model convolved with the
output synthesized beam, and a fidelity image. (see ALMA memo 398 for
description of fidelity, which is approximately the output image
divided by the difference between input and output)

%% The combined task {\tt simdata} is modular: one can
%% modify one's sky model, predict visibilities, corrupt the Measurement
%% Set, re-image, and analyze the result all separately, provided in a
%% few cases the filenames are set correctly.  
%%%%%%%%%%%%%%%%%%%%%%%%%%%%%%%%%%%%%%%%%%%%%%%%%%%%%%%%%%%%%%%%%
%%%%%%%%%%%%%%%%%%%%%%%%%%%%%%%%%%%%%%%%%%%%%%%%%%%%%%%%%%%%%%%%%
\section{Simulating ALMA observations with {\tt simalma}}
\label{section:sim.simalma}

The task {\tt simalma} simulates an ALMA observation by ALMA 12-m, ACA-7m
and total power arrays.
It takes an input model image or a list of components, plus
configurations of ALMA antennas (locations and sizes), and simulates a
particular ALMA observation (specified by mosaic setup and
observing cycles and times).  The outputs are measurement sets.
The task optionally generates synthesized images from the measurement
sets as {\tt simanalyze} does. 

Technically speaking, {\tt simalma} internally
calls {\tt simobserve} and {\tt simanalyze} as many times as necessary
to simulate and analyze an ALMA observation.
Some of the simulation ({\tt simobserve}) and imaging 
({\tt simanalyze}) parameters are automatically set to values typical
of ALMA observations in {\tt simalma} (see
\S~\ref{section:sim.simalma.casa410} for more details). 
Thus, it has a simpler task interface compared to {\tt simobserve} plus
{\tt simanalyze} at the cost of limited flexibility. 
If you want to have more control on simulation setup, it is available
by manually running {\tt simobserve} and {\tt simanalyze} multiple
times or by using {\tt sm} tools.


The {\tt simalma} inputs are:
\small
\begin{verbatim}
project             =         ''        #  root prefix for output file names
project             =      'sim'        #  root prefix for output file names
dryrun              =      False        #  dryrun=True will only
                                        #  produce the informative report, not run
                                        #   simobserve/analyze
skymodel            =         ''        #  model image to observe
     inbright       =         ''        #  scale surface brightness of
                                        #   brightest pixel e.g. "1.2Jy/pixel"
     indirection    =         ''        #  set new direction
                                        #   e.g. "J2000 19h00m00 -40d00m00"
     incell         =         ''        #  set new cell/pixel size e.g. "0.1arcsec"
     incenter       =         ''        #  set new frequency of center
                                        #  channel e.g. "89GHz" (required even
                                        #   for 2D model)
     inwidth        =         ''        #  set new channel width
                                        #  e.g. "10MHz" (required even for 2D model)

complist            =         ''        #  componentlist to observe
     compwidth      =     '8GHz'        #  bandwidth of components

setpointings        =       True        
     integration    =      '10s'        #  integration (sampling) time
     direction      =         ''        #  "J2000 19h00m00 -40d00m00"
                                        #   or "" to center on model
     mapsize        =   ['', '']        #  angular size of map or "" to cover model

graphics            =     'both'        #  display graphics at each
                                        #   stage to [screen|file|both|none]
verbose             =      False        
overwrite           =      False        #  overwrite files starting with $project
async               =      False        #  If true the taskname must
                                        #   be started using simalma(...)
\end{verbatim}
%% $
\normalsize

The task {\tt simalma} is designed as a task that is invoked only once for
a simulation setup. It always sets up skymodel and pointings.
That means that {\tt simalma} is not supposed to be run multiple times for a
project, unlike {\tt simobserve} and {\tt simanalyze}. 
The task {\tt simalma} may ignore or overwrite the old results when it is
run more than once with the same project name. 

There are options in {\tt simalma} to simulate
observation of ACA 7-m and total power arrays, to apply thermal noise, and/or
to generate images from simulated measurement sets.   One inputs a vector of configurations, and a corresponding vector of {\tt totaltime}s to observe each component. 
Thermal noise is added to visibilities when $ {\tt pwv} \, > \, 0 $.
The ATM atmospheric model is constructed from the characteristics of
the ALMA site and a user defined Precipitable Water Vapour ({\tt pwv})
value. 
Set $ {\tt pwv} \, = \, 0 $ to omit the thermal noise. 
Finally, when {\tt image = True}, synthesized images are generated
from the simulated measurement sets.

\subsection{Implementation details}
\label{section:sim.simalma.casa410}

As mentioned in the previous section, {\tt simalma} automatically sets
some of the simulation and imaging parameters to values typical of ALMA
observations. The implementations of antenna configurations, pointings,
integration time, and imaging in CASA 4.2.0 are described in this
section.

\smallskip
{\bf Antenna Configuration}:

The configurations of the ALMA 12-m and 7-m arrays are defined by the
{\tt antennalist} parameter, which can be a vector.  Each element of the vector can be either the name of an antenna
configuration file or a desired resolution, e.g., {\tt
  `alma;cycle1;5arcsec'}.  Some examples:
\begin{itemize}
\item {\tt antennalist = ['alma.cycle2.5.cfg','aca.cycle2.i.cfg']};  {\tt totaltime = ['20min','2h']'}: Will observe the 12-m array in configuration C32-5 for 20 minutes and the ACA 7-m array for 2 hours.
\item {\tt antennalist = ['alma;cycle2;0.5arcsec','aca.i.cfg']};  {\tt totaltime = ['20min','2h']'}: Will observe the 12-m array in whatever cycle 2 configuration yields a zenith synthesized beam as close as possible to 0.5 arcsec (at the center frequency of your skymodel) for 20 minutes and the ACA 7-m array for 2 hours.
\item {\tt antennalist = ['alma.cycle1.2.cfg','aca.cycle2.i.cfg']};  {\tt totaltime = '20min'}: Will observe the 12-m array in cycle 1 configuration for 20 minutes and the ACA 7-m array for the default of 2$\times$(12-m time) = 1h20min.  This parameter setting will also generate a warning that the user is combining configurations from different ALMA Cycles (but the simulation will run despite that).
\end{itemize}

Total power can either be included along with interferometric configurations e.g.  {\tt antennalist = ['alma.cycle1.2.cfg','aca.cycle2.i.cfg','alma.tp.cfg']}, 
or by using the {\tt tpnant} and {\tt tptime} parameters.  The latter is preferred since it allows greater control (in particular the number of total power antennas to use -- if more than one is used, multiple total power observations will be generated and combined in imaging).


\medskip
{\bf Field Setup}:

There are two ways to setup pointings, i.e., Rectangle Setup and
Multi-Pointing.

In the Rectangle Setups, pointings are automatically calculated from
the pointing centre ({\tt direction}) and the map size.  A rectangular
map region is covered by a hexagonal grid ({\tt maptype = `alma'})
with Nyquist sampling, i.e., $ 0.48\, {\rm PB} $ spacing (where $ {\rm
  PB} \, \equiv \, 1.2 \, \lambda / D $), in both ALMA 12-m and ACA
7-m array simulations.  A slightly larger area is mapped in ACA total
power simulations for later combination with interferometer
visibilities. The map area is extended by $ 1 \, {\rm PB} $ in each
direction and covered by a lattice grid with $ 0.225\, {\rm PB} $
spacing.

In Multi-Pointing, a list of pointings is defined in the {\tt
  direction} parameter or read from a file (when {\tt setpointings =
  False}). The ALMA 12-m and ACA 7-m arrays observe the specified
directions.  The ACA total power simulations map either (1) square
regions of $ 2\, {\rm PB} $ extent centred at each of the pointings,
or (2) a rectangle region that covers all the pointings.  Either (1)
or (2), whichever can be done with the smaller number of points, is
selected. The pointing spacing in total power simulations is, again, $
0.225\, {\rm PB} $ in lattice grids.


It is advisable that for Total Power Simulations, the field is chosen
sufficiently large, maybe padding at least 1-2 primary beams on each
side.

%when doing total power simulations, you want to make the field as large as possible... padding with %at least 1-2 primary beams on each side is preferable. (Unless the sims are doing this automatically%now?) Maybe it would be good to include a note to indicate that simulating total power images for a %source that is small compared to this may result in strange edge effects.


\medskip
{\bf Integration time}:

The total observation time of each component or configuration is
defined by the {\tt totaltime} parameter as noted above.  A scalar
will trigger use of the Cycle 2 default time multipliers, 1:0.5:2:4
for the first 12-m configuration, any additional 12-m configurations,
any 7-m configuration, and any total power observation.

In general, the integration time (dump interval) of simulations is
defined by the {\tt integration} parameter with an exception.  Since
the ACA total power array always observes larger areas compared to the
ALMA 12-m and ACA 7-m arrays, it is possible that the ACA total power
array cannot cover all pointings in the given observation time.  In
such a case, the integration time in the total power simulation is
scaled so that the all pointings are observed at least once in its
observation time, i.e., ${\tt integration\_TP} \, = \, {\tt tptime} \,
/ \, ( ${\tt the number of total power pointings}$ ) $.


\medskip
{\bf Imaging and combination of ALMA with ACA}:

% 4.1
The CLEAN algorithm is used in {\tt simalma} to generate images
from visibilities. The visibilities are weighted to UV-plane using
Briggs weighting.

When ACA observations are simulated, visibilities of ACA 7-m are
weighted by the relative sensitivities to ALMA 12-m visibilities, 
and both data sets are concatenated before imaging. 
The relative weight of ACA 7-m visibilities is defined in proportion
to the difference of beam area, i.e., $ (7/12)^2 \, = \, 0.34 $.
This is because {\tt simalma} uses a bandwidth and an integration time
common to both ALMA 12-m and ACA 7-m simulations.

The interferometer and total power images are combined using 
{\tt feather} task when total power observations are included.
The total power image is scaled by the interferometer primary beam
coverage before combination. 
The final image product is the combined image corrected for the
interferometer primary beam coverage. The output image of the
{\tt feather} task is divided by the interferometer primary beam
coverage in the final step.

%%%%%%%%%%%%%%%%%%%%%%%%%%%%%%%%%%%%%%%%%%%%%%%%%%%%%%%%%%%%%%%%%

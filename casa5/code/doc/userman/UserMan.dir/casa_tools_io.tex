%%%%%%%%%%%%%%%%%%%%%%%%%%%%%%%%%%%%%%%%%%%%%%%%%%%%%%%%%%%%%%%%%
%%%%%%%%%%%%%%%%%%%%%%%%%%%%%%%%%%%%%%%%%%%%%%%%%%%%%%%%%%%%%%%%%
%%%%%%%%%%%%%%%%%%%%%%%%%%%%%%%%%%%%%%%%%%%%%%%%%%%%%%%%%%%%%%%%%

% STM 2007-04-13  split from previous version
% STM 2007-04-14  major rewriting to include ms, handling, export
% STM 2007-04-15  tools here
% JO  2010-10-12  edits for Release 3.1.0

\chapter{Data Import, Handling, and Export}
\label{chapter:iotool} 

%%%%%%%%%%%%%%%%%%%%%%%%%%%%%%%%%%%%%%%%%%%%%%%%%%%%%%%%%%%%%%%%%
%%%%%%%%%%%%%%%%%%%%%%%%%%%%%%%%%%%%%%%%%%%%%%%%%%%%%%%%%%%%%%%%%
\section{CASA Measurement Sets}
\label{section:iotool.ms}

Data is handled in CASA via the {\tt table} system.  In particular,
visibility data are stored in a CASA table known as a Measurement Set
(MS).  Note that images are handled through special
{\tt image} tables, although standard FITS I/O is also supported.
Images and {\tt image} data are described in a separate chapter.

Unless your data was previously processed by CASA or software based
upon its predecessor {\tt aips++}, you will need to import it into
CASA as an MS.  Supported formats include some ``standard'' flavors
of UVFITS, the VLA ``Export'' archive format, and most recently,
the ALMA Science Data Model (ASDM) format.

%%%%%%%%%%%%%%%%%%%%%%%%%%%%%%%%%%%%%%%%%%%%%%%%%%%%%%%%%%%%%%%%%
\subsection{Measurement Set Structure}
\label{section:iotool.ms.struct}

Visibility data are stored in a CASA table known as a
Measurement Set (MS).  The most recent specification for the MS
is {\bf Aips++ MeasurementSet definition version 2.0}
(\url{http://aips2.nrao.edu/docs/notes/229/229.html}).
This documentation will eventually be updated to the CASA 
document system.

An MS consists of a main table containing the
visibility data and associated sub-tables containing auxiliary or
secondary information.  Table \ref{tabselect} lists data selection
parameters which may be used during a typical data reduction session
while Table \ref{tabmain} identifies the commonly accessed components
of the {\tt MAIN} table in data sets.

Each row in a data column in the MS (e.g. {\tt DATA}, {\tt ALMA\_PHAS\_CORR},
{\tt CORRECTED\_DATA}) contains a matrix of observed complex visibilities at
a single time stamp, for a single baseline in a single spectral
window.  The shape of the data matrix is given by the number of
channels and the number of correlations (voltage-products) formed by
the correlator for an array.  

All CASA data files, including Measurement Sets, are written into
the current working directory by default, with each CASA table
represented as a separate sub-directory.  MS names therefore need only
comply with UNIX file or directory naming conventions, and can be
referred to from within CASA directly, or via full path names.

\vspace{5mm}
\begin{table}[ht]
\caption[Common Data Selection Parameters]
        {\label{tabselect} Common Data Selection Parameters}
\begin{center}
\begin{tabular}{|ll|} \hline
{\bf Parameter}   &   {\bf Contents}                  \\
\hline
\hline
ANTENNA1      &   First antenna in baseline   \\
ANTENNA2      &   Second antenna in baseline  \\
FIELD\_ID      &   Field (source no.) identification  \\
DATA\_DESC\_ID  &   Spectral window number, polarization identifier pair (IF no.)  \\
ARRAY\_ID      &   Subarray number                \\
OBSERVATION\_ID  &   Observation identification   \\
POLARIZATION\_ID  &   Polarization identification \\ 
SCAN\_NUMBER   &   Scan number                    \\
TIME          &   Integration midpoint time      \\
UVW           &   UVW coordinates                \\
\hline
\end{tabular}
\end{center}
\end{table}

\vspace{5mm}
\begin{table}[hb]
\caption[Commonly accessed {\tt MAIN} Table components for ALMA data]
        {\label{tabmain}Commonly accessed {\tt MAIN} Table components for ALMA data}
\begin{center}
\begin{tabular}{|ll|} \hline
  {\bf Column}      &  {\bf Format}        \\         
                    &  {\bf Comments}      \\        
  DATA              &  Complex(N$_c$, N$_f$)  \\       
                    &  Complex visibility matrix  \\
                    &  =ALMA\_PHASE\_CORR by default  \\
  WEIGHT\_SPECTRUM   &  Float(N$_c$)    \\            
                    &  Weight for whole data matrix \\     
  ALMA\_PHASE\_CORR   &  Complex(N$_c$, N$_f$) \\        
                    &  On-line phase corrected complex visibility matrix  \\
                    &  {\it (Not in VLA data)}  \\
  ALMA\_NO\_PHAS\_CORR  &  Bool(N$_c$, N$_f$)    \\     
                     & Complex visibility matrix that has not been phase corrected  \\
                     & {\it (Not in VLA data)}  \\
  ALMA\_PHAS\_CORR\_FLAG\_ROW  &  Bool(N$_c$, N$_f$)  \\
                           &  Flag to use phase-corrected data or not, Default=F  \\
                           &  {\it (not in VLA data)} \\    
  CORRECTED\_DATA   &  Complex(N$_c$, N$_f$)  \\     
                   &  Corrected data created by calibrater or imager tools \\ 
  MODEL\_DATA       &  Complex(N$_c$, N$_f$)      \\ 
                   &  Model data created by calibrater or imager tools  \\
  FLAG             &  Bool(N$_c$, N$_f$) \\
                   &  cumulative data flags    \\
\hline
\end{tabular}
\end{center}
\end{table}
{\bf Note:} when you examine table entries like {\tt FIELD\_ID} or 
{\tt DATA\_DESC\_ID} with the table browser, you will see 0-based numbers.

\begin{figure}[ht]
\gname{jbrowser_ms}{6}
\caption{\label{fig:tablekeyword} The structure of a Measurement
  Set. The tables which compose a Measurement Set named ngc5921.ms on
  disk.}
\hrulefill
\end{figure}
 
\begin{figure}[h]
\gname{jbrowser1}{6}
\caption{\label{fig:tablekeyword2} A few of the {\tt MAIN} table columns in a
  Measurement Set.} 
\hrulefill
\end{figure}
 
\begin{figure}[hb]
\gname{B0319_uvcoverage}{5}
\caption{\label{fig:matplotlib2} matplotlib plotter. The buttons on the
  lower left are: 1,2,3) {\bf Home, Back and Forward}. Click to navigate
  between previously defined views (akin to web navigation), 4)
  {\bf pan}. Click and drag to pan to a new position, 5) {\bf zoom}. Click to
  define a rectangular region for zooming, 6) {\bf Subplot
  Configuration}. Click to configure the parameters of the subplot and
  spaces for the figures, 7) {\bf Save}. Click to launch a file save
  dialog box. The cursor readout is on the bottom right.}
\hrulefill
\end{figure}

%%%%%%%%%%%%%%%%%%%%%%%%%%%%%%%%%%%%%%%%%%%%%%%%%%%%%%%%%%%%%%%%%
%%%%%%%%%%%%%%%%%%%%%%%%%%%%%%%%%%%%%%%%%%%%%%%%%%%%%%%%%%%%%%%%%



%%%%%%%%%%%%%%%%%%%%%%%%%%%%%%%%%%%%%%%%%%%%%%%%%%%%%%%%%%%%%%%%%
\section{UVFITS Import and Export}
\label{section:iotool.uvfits}

Import and export methods include {\tt ms.fromfits} and
{\tt ms.tofits}.


%%%%%%%%%%%%%%%%%%%%%%%%%%%%%%%%%%%%%%%%%%%%%%%%%%%%%%%%%%%%%%%%%
%%%%%%%%%%%%%%%%%%%%%%%%%%%%%%%%%%%%%%%%%%%%%%%%%%%%%%%%%%%%%%%%%

\chapter{Writing Glish Functions\label{Useful.GlishFunctions}}

\textit{Users should consult \textbf{\htmladdnormallink{The Glish User's Manual}
{http://www.cv.nrao.edu/aips++/glish/manual/}}
for general glish programming.}

Writing glish functions is very useful for assembling \aipspp
tools.  Utilities checked into the system should follow a set 
guidelines.  

\section{Guidelines}

As a glish tool writer it helps to distinguish between private member functions
and global functions (This will help avoid name-space confilicts
with other scripts that might be included).
Write private functions as a record  i.e.  app.function := function
At the end of the script declare the record const to make sure a user
doesn't accidently overwrites it.

Glish functions intended for user interaction need both
programmer and user documentation.  Functions not visible to users
need programmer documentation similar to the C++/C function format.

Glish functions that users interact with are documented using "Help Atoms"
(See Paul Shannon's
note 187 for More details).  Essentially you specify a series of attributes
that describe the function.
All global functions in the system library must have the help atoms defined!
Please follow the glish script template.
Until we implement code copping on \aipspp scripts we need to exercise
self-discipline.

Include glish code only once.  How do you do this?  Put the glish code 
inside an "include" guard.  An example follows:
\begin{verbatim}
if(!is_defined("myGlishScript_included")){
   myGlishScript_include := 'yes';
   ....
}
\end{verbatim}

Now how do we prevent the clients starting 
before we really need them?  Write a function similar to the
start\_myClient example shown below.

\begin{verbatim}
const start_myClient := function()
{
   if(!is_defined("myClient_started")){
      myClient_started := 'yes'
      global myClient := client("my");
   }
}
\end{verbatim}
You call start\_myClient in your glish script.  It will only execute if the
variable myClient\_started has not been defined.


\section{Conventions}

Functions are define as lowercase with underscores between words.

Variables are mixed upper and lowercase with the first letter always lower
case.
Use descriptive variable names.  It will help others understand your code.

Use comments when needed and lots of whitespace to make things readable!

\section{Pitfalls and Troubles}

\begin{itemize}
\item Avoid terminating glish clients.
\end{itemize}

\section{Glish Template}
Please use the following template to write your glish scrips.  It's available
from code/install/codedevel/template-glish (or will be shortly).
\begin{verbatim}

Text enclosed in <> needs to be changed to the appropriate function name.

------------cut here--------------------------

# <GlishFunction.g>: short description ...
# Copyright (C) 1996
# Associated Universities, Inc. Washington DC, USA.
#
# This library is free software; you can redistribute it and/or modify it
# under the terms of the GNU Library General Public License as published by
# the Free Software Foundation; either version 2 of the License, or (at your
# option) any later version.
#
# This library is distributed in the hope that it will be useful, but WITHOUT
# ANY WARRANTY; without even the implied warranty of MERCHANTABILITY or
# FITNESS FOR A PARTICULAR PURPOSE.  See the GNU Library General Public
# License for more details.
#
# You should have received a copy of the GNU Library General Public License
# along with this library; if not, write to the Free Software Foundation,
# Inc., 675 Massachusetts Ave, Cambridge, MA 02139, USA.
#
# Correspondence concerning AIPS++ should be addressed as follows:
#        Internet email: aips2-request@nrao.edu.
#        Postal address: AIPS++ Project Office
#                        National Radio Astronomy Observatory
#                        520 Edgemont Road
#                        Charlottesville, VA 22903-2475 USA
#

# Create an include guard

if(! is_defined("<GlishFunction_g_included>")) {
  <GlishFunction_g_included> := 'yes'

#
#
# include "other.g" files as needed
#

#
#Define glish functions (Remove const while debugging otherwise...)
#

const <GlishFunction> := GlishFunction(arg...) {  
   .... }


} # Terminates the include guard


\end{verbatim}
